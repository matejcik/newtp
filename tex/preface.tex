\chapter*{Introduction}
\addcontentsline{toc}{chapter}{Introduction}

Today, in the age of ubiquituous networking, it is as important as ever to be able to share files between
devices. We have smart cameras and phones, home fileservers and shared document storages etc., and we want to
work with data stored on these devices. Moreover, one of the basic requirements of cloud computing, which is
rising in popularity and importance, is the ability of clients to efficiently and securely access their data
that reside on remote servers.

This means that networking protocols on connected devices play a huge role in perceived performance, utility
and feature set of those devices and networks - especially when we're talking about low-powered devices,
unreliable wireless networks and other non-optimal conditions. However, details and properties of those
protocols are abstracted away from their users. The reason for this is obvious: non-technical users are
interested in features and performance of concrete applications, not in implementation details. But this means
that, for instance, actual low-level security and resilience to attacks remains hidden. Also, less and less
attention is paid to cross-application and cross-platform interoperability.

We decided to examine one of the lower levels in the protocol hierarchy - basic filesystem-like access to
files and directory structures. We know from user experience that existing solutions, while sufficient, are
far from perfect in real-world conditions. Interoperability between different implementations is less than ideal,
and some protocols are better able to cope with degradation in network environment (packet loss, high latency)
than others.

We decided to design a new, modern file transfer protocol, in an attempt to examine possible solutions to
various problems and challenges. We hope to find out how tradeoffs in the design affect different application
scenarios, how much protocol efficiency depends on usage by applications and how features of a protocol relate
to its vulnerability to attacks. This could allow us to come up with a protocol that is better suited for some
usecases than the protocols available today.

The results of this work are three-fold. First and foremost, it is the protocol design, along with reference
implementations for server and client. Second, the thesis describes in detail the various properties of a file
sharing protocol, available design choices and tradeoffs, and how design process works from the ground up.
This will be useful for people designing a protocol best suited for some particular usecase. And thirdly, we
present a summary evaluation of relevant characteristics of protocols that are widely in use today.
