% vim: tw=110:fo=an1lt

\chapter{Related works}

This chapter examines several popular existing protocols and general characteristics
of their implementations.

TODO tcp/udp/whatever

\section{FTP}

FTP, the File Transfer Protocol, is one of the oldest network protocols for file transfer, and probably the
oldest that is still widely used. It was first introduced in~1973 as RFC~454~\cite{rfc454} and the most recent
accepted standard is RFC~959~\cite{rfc959} from~1985. Several proposed extensions were published after that,
mostly describing how the standard has evolved in practical use.

With FTP, the client establishes a text-based, line-oriented {\it control} connection to the server. Through this
the user can configure session parameters, traverse and list directories and manipulate directory entries,
similar to an interactive shell. Many commands are designed for human use, e.g. the HELP command, or the LIST
command that lists contents of the current directory in a platform-dependent textual format. Actual file
transfers run on separate {\it data} connections which are negotiated on the control connection. This means that
in order to support FTP transparently, firewalls and NATs must understand the FTP protocol.

TODO zmínit že řeší spoustu nepodstatných kravin

Users authenticate themselves with user name, password and optionally an account name. The protocol specifies
no encryption, so all information is transferred in plain text. RFC~4217~\cite{rfc4217} specifies a variant of
FTP that uses TLS for encryption. However, this makes FTP traffic opaque, so special care must be taken when
firewalls or NATs are involved.

\section{HTTP}

Mentioned mostly for completeness, HTTP (Hyper-Text Transfer Protocol) is at its core a protocol for
retrieving files from remote servers. Although it has limited capabilities for creating, updating and deleting
content, those are usually used in application-specific manner and not for general file manipulation. It
doesn't support directory manipulation at all. Current HTTP specification is described by
RFC~2616~\cite{rfc2616} from~1999.

HTTP is text-based. Each client request specifies a method (one of GET, PUT, POST or DELETE) and an URL on the
first line, followed by any number of headers, one per line.  Headers can specify content metadata,
instructions for proxies, client expectations etc. The headers section can be followed by a body containing
arbitrary data. Similarly, server response starts with a numeric response code, followed by headers and data
payload.

The protocol is fully stateless and has no concept of sessions beyond single request/response exchanges,
although it allows reusing a single TCP connection as an optimization.  RFC~2069~\cite{rfc2069} specifies a
simplistic authentication model, based on username/password combinations. RFC~2818~\cite{rfc2818} describes
how to use the protocol with TLS, which is commonly known as HTTPS.

\subsection{WebDAV}

WebDAV (Web Distributed Authoring and Versioning) protocol is an extension of HTTP, designed to provide the
World-Wide Web network with standardized authoring capabilities. It was first specified in~1999 as
RFC~2518~\cite{rfc2518}, with RFC~4918~\cite{rfc4918} from 2007 being the latest published specification.
As~of~2013, WebDAV and its extensions are actively maintained by a number of IETF working groups.

WebDAV specifies several new methods that add directory manipulation and locking functionality, as well as
corresponding headers and XML-based formats of data payloads. It also specifies detailed semantics for file
and directory operations. Despite its name, versioning is not part of WebDAV's core functionality - it is
specified as an extension \cite{rfc3253}.

WebDAV does not modify the way HTTP operates, so it is still stateless with no session support. It uses HTTP's
authentication models and it can run under HTTPS encryption mode. It is also backwards-compatible with
HTTP-only proxies. Depending on the nature of the proxy, WebDAV traffic can either pass through unchanged, or
a failure is detected and doesn't endanger consistency of server state.

The whole of WebDAV's functionality is relatively complex: it concerns itself with issues like authorship,
conflict resolution, locking policies, content properties and metadata etc. Also, it is designed to run on
WWW, which means that it has to care about unique URLs, many users, anonymous vs. authorized access, proxying
and so on.  So while client implementations can be relatively simple (this is even a requirement of the
original WebDAV proposal), servers tend to be complicated.

WebDAV is natively supported in most modern OSes, but due to the complexity, implementations tend to be
slightly incompatible.

\section{SMB}

Server Message Block is an evolution of a proprietary networking protocol designed at IBM around the year
1985. Originally using NetBIOS API to abstract away the transport layer, its current version usually runs
directly on TCP/IP.  The most recent version is SMB 3.0, which first appeared in 2012 with Windows 8 and is
actively maintained by Microsoft. It is specified in~\cite{mssmb2} as part of Microsoft's Open Specifications
initiative.

SMB was originally designed as an all-encompasing networking protocol for small local networks. Apart from
publishing parts of directory trees, it allows sharing printers, ports and even some IPC mechanisms. The
NetBIOS API allows service discovery, but these days this capability is mostly superseded by use of
Dynamic~DNS. SMB is often used as a filesystem, e.g. for storing user profile data on a central server, its
directory access features are filesystem-like and commands roughly correspond to filesystem access primitives.

The protocol is binary. It specifies a packet format and a number of commands, their arguments and responses.
Since SMB 2.0, GSS-API is used for user authentication. Support for digitally-signed communications existed in
older versions of the protocol, and version 3.0 enhances that with full encryption support.

\section{NFS}

TODO

\section{SFTP}

TODO

\section{rsync}

jenom v krátkosti zmínit že to existuje a že to je specializovaný

\section{Others}

this section mentions that the list here is by no means exhaustive
