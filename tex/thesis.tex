%%% Hlavní soubor. Zde se definují základní parametry a odkazuje se na ostatní části. %%%

%% Verze pro jednostranný tisk:
% Okraje: levý 40mm, pravý 25mm, horní a dolní 25mm
% (ale pozor, LaTeX si sám přidává 1in)
\documentclass[12pt,a4paper]{report}
\setlength\textwidth{145mm}
\setlength\textheight{247mm}
\setlength\oddsidemargin{15mm}
\setlength\evensidemargin{15mm}
\setlength\topmargin{0mm}
\setlength\headsep{0mm}
\setlength\headheight{0mm}
% \openright zařídí, aby následující text začínal na pravé straně knihy
\let\openright=\clearpage

%% Pokud tiskneme oboustranně:
% \documentclass[12pt,a4paper,twoside,openright]{report}
% \setlength\textwidth{145mm}
% \setlength\textheight{247mm}
% \setlength\oddsidemargin{15mm}
% \setlength\evensidemargin{0mm}
% \setlength\topmargin{0mm}
% \setlength\headsep{0mm}
% \setlength\headheight{0mm}
% \let\openright=\cleardoublepage

%% Použité kódování znaků: obvykle latin2, cp1250 nebo utf8:
\usepackage[utf8]{inputenc}

%% Ostatní balíčky
\usepackage{graphicx}
\usepackage{amsthm}
\usepackage{enumitem}

%% Balíček hyperref, kterým jdou vyrábět klikací odkazy v PDF,
%% ale hlavně ho používáme k uložení metadat do PDF (včetně obsahu).
%% POZOR, nezapomeňte vyplnit jméno práce a autora.
\usepackage[ps2pdf,unicode]{hyperref}   % Musí být za všemi ostatními balíčky
\hypersetup{pdftitle=File sharing protocol}
\hypersetup{pdfauthor=Jan Matějek}

%%% Drobné úpravy stylu

% Tato makra přesvědčují mírně ošklivým trikem LaTeX, aby hlavičky kapitol
% sázel příčetněji a nevynechával nad nimi spoustu místa. Směle ignorujte.
\makeatletter
\def\@makechapterhead#1{
  {\parindent \z@ \raggedright \normalfont
   \Huge\bfseries \thechapter. #1
   \par\nobreak
   \vskip 20\p@
}}
\def\@makeschapterhead#1{
  {\parindent \z@ \raggedright \normalfont
   \Huge\bfseries #1
   \par\nobreak
   \vskip 20\p@
}}
\makeatother

% Toto makro definuje kapitolu, která není očíslovaná, ale je uvedena v obsahu.
\def\chapwithtoc#1{
\chapter*{#1}
\addcontentsline{toc}{chapter}{#1}
}

\begin{document}

% Trochu volnější nastavení dělení slov, než je default.
\lefthyphenmin=3
\righthyphenmin=3

%%% Titulní strana práce

\pagestyle{empty}
\begin{center}

\large

Charles University in Prague

\medskip

Faculty of Mathematics and Physics

\vfill

{\bf\Large BACHELOR THESIS}

\vfill

\centerline{\mbox{\includegraphics[width=60mm]{logo.eps}}}

\vfill
\vspace{5mm}

{\LARGE Jan Matějek}

\vspace{15mm}

% Název práce přesně podle zadání
%{\LARGE\bfseries File sharing protocol}
%{\LARGE\bfseries A file sharing protocol}
{\LARGE\bfseries Protokol pro sdílení souborů}

\vfill

% Název katedry nebo ústavu, kde byla práce oficiálně zadána
% (dle Organizační struktury MFF UK)
Department of Applied Mathematics

\vfill

\begin{tabular}{rl}

Supervisor of the bachelor thesis: & Martin Mareš \\
\noalign{\vspace{2mm}}
Study programme: & Computer Science \\
\noalign{\vspace{2mm}}
Specialization: & Programming \\
\end{tabular}

\vfill

% Zde doplňte rok
Prague 2013

\end{center}

\newpage

%%% Následuje vevázaný list -- kopie podepsaného "Zadání bakalářské práce".
%%% Toto zadání NENÍ součástí elektronické verze práce, nescanovat.

%%% Na tomto místě mohou být napsána případná poděkování (vedoucímu práce,
%%% konzultantovi, tomu, kdo zapůjčil software, literaturu apod.)

\openright

\noindent
%Dedication.
{\bf Acknowledgements}\\
I would like to thank my supervisor, Mgr. Martin Mareš, Ph.D., for his help and support
with writing this work, and especially for his patience and immeasurable tolerance for my
irregular working schedule.\\
Secondly, I want to thank my friends Andy and Klára. This work would not exist without
their support and encouragement.

\newpage

%%% Strana s čestným prohlášením k bakalářské práci

\vglue 0pt plus 1fill

\noindent
I declare that I carried out this bachelor thesis independently, and only with the cited
sources, literature and other professional sources.

\medskip\noindent
I understand that my work relates to the rights and obligations under the Act No.
121/2000 Coll., the Copyright Act, as amended, in particular the fact that the Charles
University in Prague has the right to conclude a license agreement on the use of this
work as a school work pursuant to Section 60 paragraph 1 of the Copyright Act.

\vspace{10mm}

\hbox{\hbox to 0.5\hsize{%
In ........ date ............
\hss}\hbox to 0.5\hsize{%
signature of the author
\hss}}

\vspace{20mm}
\newpage

%%% Povinná informační strana bakalářské práce

\vbox to 0.5\vsize{
\setlength\parindent{0mm}
\setlength\parskip{5mm}

Název práce:
Protokol pro sdílení souborů
% přesně dle zadání

Autor:
Jan Matějek

Katedra:  % Případně Ústav:
Katedra aplikované matematiky
% dle Organizační struktury MFF UK

Vedoucí bakalářské práce:
Mgr. Martin Mareš, Ph.D., Katedra aplikované matematiky
% dle Organizační struktury MFF UK, případně plný název pracoviště mimo MFF UK

Abstrakt:
% abstrakt v rozsahu 80-200 slov; nejedná se však o opis zadání bakalářské práce
Dnešní síťová výpočetní prostředí silně spoléhají na možnost efektivně přesouvat data mezi počítači. Abychom
porozuměli problémům a úlohám týkajícím se sdílení souborů a publikování adresářových stromů v počítačové
síti, prozkoumali a shrnuli jsme vlastnosti několika populárních protokolů. S použitím získaných znalostí jsme
vytvořili NewTP, moderní protokol pro sdílení souborů, vhodný pro použití v Internetu. Zvláštní důraz byl
kladen na jednoduchost implementace a odolnost proti útokům. Prokázali jsme reálnou použitelnost protokolu
tím, že jsme naimplementovali referenční server a modul souborového systému FUSE.

Klíčová slova:
přenos souborů, síťový souborový systém, síťový protokol
% 3 až 5 klíčových slov

\vss}\nobreak\vbox to 0.49\vsize{
\setlength\parindent{0mm}
\setlength\parskip{5mm}

Title:
A file sharing protocol
% přesný překlad názvu práce v angličtině

Author:
Jan Matějek

Department:
Department of Applied Mathematics
% dle Organizační struktury MFF UK v angličtině

Supervisor:
Mgr. Martin Mareš, Ph.D., Department of Applied Mathematics
% dle Organizační struktury MFF UK, případně plný název pracoviště
% mimo MFF UK v angličtině

Abstract:
% abstrakt v rozsahu 80-200 slov v angličtině; nejedná se však o překlad zadání bakalářské práce
Today's networked computing environments rely heavily on the ability to efficiently transfer data between
computers. In order to understand the issues and challenges involved in sharing files and publishing directory
trees over a computer network, we examined and summarized features of a number of popular protocols. Using
this knowledge, we designed NewTP, a modern file sharing protocol suitable for the Internet. Special emphasis
was placed on simplicity of implementation and resistance to attacks. We have proven the protocol's real-world
viability by implementing a reference server and a FUSE filesystem module.

Keywords:
file transfer, network filesystem, network protocol
% 3 až 5 klíčových slov v angličtině

\vss}

\newpage

%%% Strana s automaticky generovaným obsahem bakalářské práce. U matematických
%%% prací je přípustné, aby seznam tabulek a zkratek, existují-li, byl umístěn
%%% na začátku práce, místo na jejím konci.

\openright
\pagestyle{plain}
\setcounter{page}{1}
\tableofcontents

%%% Jednotlivé kapitoly práce jsou pro přehlednost uloženy v samostatných souborech

\chapter*{Introduction}
\addcontentsline{toc}{chapter}{Introduction}

Today, in the age of ubiquituous networking, it is as important as ever to be able to share files between
devices. We have smart cameras and phones, home fileservers and shared document storages etc., and we want to
work with data stored on these devices. Moreover, one of the basic requirements of cloud computing, which is
rising in popularity and importance, is the ability of clients to efficiently and securely access their data
that reside on remote servers.

This means that networking protocols on connected devices play a huge role in perceived performance, utility
and feature set of those devices and networks - especially when we're talking about low-powered devices,
unreliable wireless networks and other non-optimal conditions. However, details and properties of those
protocols are abstracted away from their users. The reason for this is obvious: non-technical users are
interested in features and performance of concrete applications, not in implementation details. But this means
that, for instance, actual low-level security and resilience to attacks remains hidden. Also, less and less
attention is paid to cross-application and cross-platform interoperability.

We decided to examine one of the lower levels in the protocol hierarchy - basic filesystem-like access to
files and directory structures. We know from user experience that existing solutions, while sufficient, are
far from perfect in real-world conditions. Interoperability between different implementations is less than ideal,
and some protocols are better able to cope with degradation in network environment (packet loss, high latency)
than others.

We decided to design a new, modern file transfer protocol, in an attempt to examine possible solutions to
various problems and challenges. We hope to find out how tradeoffs in the design affect different application
scenarios, how much protocol efficiency depends on usage by applications and how features of a protocol relate
to its vulnerability to attacks. This could allow us to come up with a protocol that is better suited for some
usecases than the protocols available today.

The results of this work are three-fold. First and foremost, it is the protocol design, along with reference
implementations for server and client. Second, the thesis describes in detail the various properties of a file
sharing protocol, available design choices and tradeoffs, and how design process works from the ground up.
This will be useful for people designing a protocol best suited for some particular usecase. And thirdly, we
present a summary evaluation of relevant characteristics of protocols that are widely in use today.


% vim: tw=110:fo=an1lt

\chapter{Problem Definition}

In this chapter, we present an overview of the problem area and a description of what exactly we are solving.
It should also serve as an introduction for readers unfamiliar with the topic, providing explanations of terms
that we are using throughout this work.


\section{Files and filesystems}

Most modern operating systems share a common idea of what constitutes a filesystem. It is a collection of {\it
files}, which are opaque strings of binary data. Individual files are referenced through a hierarchical
tree-like structure of {\it directories}.  You can identify any file or directory by its {\it path}, which is
a string of entry names delimited by path separator characters\footnotemark[1]. You follow the (absolute) path
by starting at the root directory and always descending into the directory named by the next path element,
until the last element is remaining, naming the target entry.

We want to be able to publish one or more subtrees of the filesystem on one computer, and access this part of
the structure from another computer over a network.


\section{Protocols}

A protocol, simply put, is a communication script for two or more parties. It describes the conversation
structure, semantics and syntax of messages. Here is an overview of several issues that must be solved by
every protocol.


\subsection{Sessions}

The most basic "protocol" imaginable is one computer just firing data packets at another, silently hoping that
the other will reassemble and understand them. We could, in theory, use this sort of protocol for transfering
files, but it is obviously very impractical. All but the most basic protocols will first establish a session,
in order to have a context for further communication. Extent of the session can vary, from simply marking
a collection of packets as belonging together, to providing ordered delivery, to authenticating parties and
encrypting transmitted data.

Basic session functionality is often provided by the transport layer, specifically the Transmission Control
Protocol (TCP). This allows the communicating parties to establish a two-way stream of octets, with
reliability, ordered delivery and flow control. Unless noted otherwise, protocols described in this work are
built on top of TCP.


\subsection{Encoding}

Protocols consist of messages, which must be encoded for over-the-wire transfer. There are two main ways to do
this: either as human-readable text, or as custom binary data.

\subsubsection{Text-based protocols}

In text-based protocols, each message is a string of words and/or numbers, nowadays usually encoded as ASCII
or UTF-8. Many such protocols also use line separator characters to delimit messages. The main advantage of
this approach is that it is readable (and writable) by humans without specialized debugging tools - for
instance, it is possible to communicate with the other party "by hand", without software implementing the
protocol. Historically, there was also the fact that two communicating systems only had to agree on text
encoding, and not about in-memory format of numbers or other data structures.

Drawbacks include overhead of text encoding, both in terms of size and processing time spent encoding and
decoding the textual representation. Another problem is that lengths of message fields are not fixed, so
parsing and framing messages is more difficult.

\subsubsection{Binary protocols}

The alternative to text-based messages is using a format similar to in-memory encoding of data. Each message
can be described by, and parsed as, a C struct\footnotemark[2]. This results in shorter messages and simpler
encoding. If the in-memory encoding is the same as in the protocol, encoding and decoding can even be skipped
completely. Parsing and framing is trivial - every kind of message can either have a fixed number of
fixed-size fields, or it can have a fixed-size part describing lengths of all variable-size fields.

Care must be taken when communicating between systems that differ in endianness.  The protocol either has to
prescribe the byte order, or it must provide means for the parties to negotiate.

Unlike text-based protocols, debugging is not straightforward, humans require specialized tools that convert
the binary messages to understandable text.


\subsection{Roles}

We recognize two basic roles for communicating parties: {\it client} and {\it server}. Client is the party
that sends requests - specifies what is to be done - and server's task is to fulfil the requests and report on
results.

Most protocols are modelled as client-server, with roles of the parties fixed in advance. Typically, the
server is listening for incoming connections and clients connect at their convenience. In a peer-to-peer
model, after establishing the connection, parties can act as both clients and servers at the same time,
sending requests to each other.

In this work, we use the following convention: a {\it server} will be the computer that contains a "master
copy" of a filesystem, and publishes a subtree of this filesystem on the network. A {\it client} is the
computer that wants to manipulate this published tree using the protocol.


%%%

\section{Capabilities of a file transfer protocol}

\subsection{Identifying files}

In order to accomplish anything useful with a protocol, there must be a way to specify on which file you want
to operate. This alone poses a number of issues that must be resolved.

The first question is whether all specified paths are absolute, relative to a fixed path, or relative to the
session's current directory.

TODO

\subsection{Download}

Downloading, i.e. transfering files from server to client, is arguably the most basic capability provided by
a file transfer protocol. Client will request a file and server will respond by sending a binary stream of the
file's contents. Failure modes are obvious: the identifier does not point to a file; file exists but cannot be
read; or this client does not have permission to access the file.

In basic form of download, the server will send full contents of the file as a continuous stream, possibly
preceded by length of the stream. However, it is often more useful to specify a range for reading. This allows
resuming of interrupted downloads, seeking in media streams and the like. It is also very useful if theclient
wants to implement filesystem-like behavior over the protocol.

A general-purpose protocol will treat files as opaque streams. However, in a specialized protocol, it might be
useful to understand structure of the file. This can allow the client to e.g. query contents of the file as
a database, download only thumbnail or embedded metadata etc.

\subsection{Upload}

Transfering from client to server is more complicated. In addition to checking for permissions, the server
must ensure that there is enough disk space, resolve issues with overwriting existing files, as well as
gracefully handle concurrent modifications. There might also be issues with missing path components - trying
to create file in a directory that does not exist. This results in many new failure modes that must be
expressed in the protocol. However, at this level, it is sufficient to add new error codes, or specify that
a single error code covers many different types of failure.

\subsection{Directories}

Although it might be tempting to handle directories as a special case of files, directories are very
different. Where files are opaque blobs, directories are inherently structured, so the protocol must represent
an API for manipulating the structure. A general-purpose protocol should provide at least the following
operations:

\begin{itemize}
	\item list contents of directory
	\item create file/directory
	\item delete file/directory
	\item rename a directory entry
\end{itemize}

\subsection{Filenames}

Different platforms have different conventions for filenames and paths. Some characters might be invalid in
file and path names, lengths of path components or the total length of path might be limited. Protocols should
either abstract away the differences (an inherently difficult problem), provide means to query platform
limitations, or at least acknowledge failure modes for path names and note that they are platform-dependent.

Another problem is correspondence between directory listings and results of operations. On Windows, filenames
are not case-sensitive, so creating file "foo" will fail if the directory already contains file "FOO".
Mac~OS~X normalizes unicode representations of filenames, which can mean that a newly created file will not
appear in subsequent directory listing, if the chosen filename was not in the corresponding normalized form.

\subsection{Filename character encoding}

One place where a file transfer protocol cannot avoid character encoding issues is in path specifiers and
directory entry names. The problem is that different systems have different conventions for filename encoding.
A protocol could specify that filenames are transfered as "plain data", but this would only push the problem
down to individual implementers. A better solution is to either prescribe or negotiate a transfer encoding,
and then transcode filenames in client and server software.

Newer filesystems enforce that a filename is a valid Unicode string, but on many widely used filesystems,
filename can be an arbitrary string of bytes\footnotemark[3]. In effect, a filesystem can store names that
cannot be properly decoded in the system's character set. This presents a problem for the transcoding process.
An obvious solution is to ignore invalid filenames, but that is arguably not very useful. It is seemingly
better to use replacement characters for the undecodable sequences, but in practice, this does not help very
much, as this way of encoding is not roundtrip-safe. The client can get a listing that contains invalid
filenames, but cannot specify them, because files with the "fixed" names do not actually exist on the server.
Moreover, it is impossible for the client to tell apart invalid filenames from valid ones that happen to
contain the replacement characters, unless the replacement characters are forbidden in valid names.

\subsection{Renaming and moving}




%%%

\footnotetext[1]{On UNIX-like systems, the path separator is a forward-slash, '/'. Windows recognizes both
forward-slash and a backslash, but backslash is used more commonly. Nevertheless, we will be using
forward-slash as a path separator everywhere.}

\footnotetext[2]{This does not mean that you can create a message by dumping the memory representation of
a corresponding C struct directly. On modern systems, fields of a struct will be memory-aligned and will
contain some amount of padding. This can differ between systems, compilers, and can even depend on particular
compiler flags.}

\footnotemark[3]{Unless the byte represents a forbidden character, e.g. forward-slash or null byte on UNIX,
shell expansion characters on Windows etc.}

% vim: tw=110:fo=an1lt

\chapter{Related works}

This chapter examines several popular existing protocols and general characteristics of their implementations.
For each protocol, we briefly describe its history, outline its features and mode of operation, and weigh its
strong and weak points, with some emphasis on security.

Unless noted otherwise, the protocols described here run on top of TCP/IP.

\section{FTP}

FTP, the File Transfer Protocol, is one of the oldest network protocols for file transfer, and probably the
oldest that is still widely used. It was first introduced in~1973 as RFC~454~\cite{rfc454} and the most recent
accepted standard is RFC~959~\cite{rfc959} from~1985. Several proposed extensions were published after that,
mostly describing how the standard has evolved in practical use.

With FTP, the client establishes a text-based, line-oriented {\it control} connection to the server. Through this
the user can configure session parameters, traverse and list directories and manipulate directory entries,
similar to an interactive shell. Commands are designed for human use first, e.g., the HELP command, or the
LIST command that lists contents of the current directory in a platform-dependent textual format. Results of
directory listing, as well as actual file transfers, run on separate {\it data} connections, which are
negotiated on the control connection. This can be used to arrange a file transfer between two servers, without
the data going through the client. It also means that in order to support FTP transparently, firewalls and
NATs must run deep packet inspection on FTP traffic.

Many of the protocol's features are designed to enable interoperation between systems with wildly different
concepts of file and directory structure. FTP supports several kinds of files depending on how the host system
stores them, several transfer modes, and explicit control of text file transfer, such as pagination and print
character conversion. These features are largely useless in the relatively homogenous computing environments
of today. On the other hand, while the protocol has distinct ASCII and EBCDIC modes, it has no support for
internationalized encodings.

FTP does not support transfer interleaving and one control connection can only support one data transfer at
a time. Furthermore, there is no support for ranged reads and writes.  It is possible to seek to a specified
point in a file before performing a data transfer, and it is possible to stop the transfer at an arbitrary
point, but this must be done in real time, and a new connection must be opened for each data transfer.  This
makes FTP particularly unsuitable for filesystem-like operation.

Users authenticate themselves with user name, password and optionally an account name. The protocol specifies
no encryption, so all information is transfered in plain text. RFC~4217~\cite{rfc4217} specifies a variant of
FTP that uses TLS for encryption. However, this makes FTP traffic opaque, so special care must be taken when
firewalls or NATs are involved.

\section{HTTP}

Mentioned mostly for completeness, HTTP (Hyper-Text Transfer Protocol) is at its core a protocol for
retrieving files from remote servers. Although it has limited capabilities for creating, updating and deleting
content, those are usually used in application-specific manner and not for general file manipulation. It
doesn't support directory manipulation at all. Current HTTP specification is described by
RFC~2616~\cite{rfc2616} from~1999.

HTTP is text-based. Each client request specifies a method (one of {\tt GET}, {\tt PUT}, {\tt POST} or {\tt
DELETE}) and an URL on the first line, followed by any number of headers, one per line.  Headers can specify
range of transfer, content metadata, instructions for proxies, client expectations etc. The headers section
can be followed by a body containing arbitrary data. Similarly, server response starts with a numeric response
code, followed by headers and data payload.

The protocol is fully stateless and has no concept of sessions beyond single request/response exchanges,
although it allows reusing a single TCP connection as an optimization. Heavy emphasis is placed on
cacheability of requests, and due to this property, it is simple to build load balancers and content delivery
networks for HTTP servers. Although less common today, proxy servers can be placed on LANs as transparent
local caches for clients retrieving the same data.

HTTP specification allows a simplistic authentication model based on username/password combinations.
RFC~2069~\cite{rfc2069} improves upon this model by introducing digest-based authentication.
RFC~2818~\cite{rfc2818} describes how to use the protocol with TLS, which is commonly known as HTTPS.

\subsection{WebDAV}

WebDAV (Web Distributed Authoring and Versioning) protocol is an extension of HTTP, designed to provide the
World-Wide Web network with standardized authoring capabilities. It was first specified in~1999 as
RFC~2518~\cite{rfc2518}, with RFC~4918~\cite{rfc4918} from 2007 being the latest published specification.
As~of~2013, WebDAV and its extensions are actively maintained by a number of IETF working groups.

WebDAV specifies several new methods that add directory manipulation and locking functionality, as well as
corresponding headers and XML-based formats of data payloads. It also specifies detailed semantics for file
and directory operations. Despite its name, versioning is not part of WebDAV's core functionality -- it is
specified as an extension \cite{rfc3253}.

WebDAV does not modify the way HTTP operates, so it is still stateless with no session support. It uses HTTP's
authentication models and it can run under HTTPS encryption mode. It is also backwards-compatible with
HTTP-only proxies. Depending on the nature of the proxy, WebDAV traffic can either pass through unchanged, or
a failure is detected and doesn't endanger consistency of server state.

The protocol's features work well for filesystem-like access, and indeed, WebDAV is often supported out of the
box in modern operating systems. However, these implementation tend to be slightly incompatible esp. in terms
of user authentication and authorship management.

\section{SMB}

Server Message Block is an evolution of a proprietary networking protocol designed at IBM around the year
1985. Originally using NetBIOS API to abstract away the transport layer, its current version usually runs
directly on TCP/IP.  The most recent version is SMB 3.0, which first appeared in 2012 with Windows 8 and is
actively maintained by Microsoft. It is specified in~\cite{mssmb2} as part of Microsoft's Open Specifications
initiative.

SMB was originally designed as an all-encompasing networking protocol for small local networks. Apart from
publishing parts of directory trees, it allows sharing printers, ports and even some IPC mechanisms. The
NetBIOS API allows service discovery, but these days this capability is mostly superseded by use of
Dynamic~DNS.

SMB is a binary protocol and its file transfer capabilities are designed for filesystem-like access. It
provides locking mechanisms and transparent failover features. Since SMB 2.0, GSS-API is used for user
authentication.  Support for digitally-signed communications existed in older versions of the protocol, and
version 3.0 enhances that with full encryption support.

As a filesystem, SMB is feature-complete and has good performance characteristics. The drawback is that due to
the complexity, it not so well suited for simpler file transfer tasks, and it is difficult to implement as
a separate protocol outside its environment.

\section{NFS}

Originally called Sun Network Filesystem, NFS was introduced in a paper~\cite{nfs85} in 1985. It was later
standardized as Network File System version 2 in RFC~1094~\cite{rfc1094}, and version 3 was released in 1995
as RFC~1813~\cite{rfc1813}. NFS can run over UDP and version 3 adds support for TCP transport layer. The
protocol itself is fully stateless, in order to improve concurrency and provide cleaner failure recovery.

NFS protocol is a neat example of modularity and technology reuse. It is defined in terms of
Sun~RPC~\cite{rfc1057} calls, the format of which is in turn specified in the XDR~\cite{rfc1832}
language.\footnote{The final on-the-wire format is binary.} The core protocol only deals with filesystem
operations, and the specification describes related protocols that provide lock management and stateful
filesystem mounting procedures. Although not formally specified in the protocol, implementations usually
contain a replay cache module~\cite{juszczak89} that reduces server load and shields the stateless server from
the non-idempotent replay problem. NFS relies on Sun~RPC security mechanisms for authentication, and version
3 does not support encryption.

NFS version 3 is now superseded by version 4. Its specification documents are still relevant, however, as they
provide great insight into file transfer protocol design and challenges.

\subsection{NFSv4}

Version 4 of the Network File System protocol was introduced in December 2000. The most up-to-date
specification is RFC~5661~\cite{rfc5661} from 2010, describing version~4.1. Version 4.2 is in active
development.

NFSv4 is the first version of NFS developed by the Internet Engineering Task Force, after Sun Microsystems
ceded formal control of the specification to the Internet Society~\cite{rfc2339}. It brings major changes to
the protocol, introducing stateful operation and mandating strong security. Other areas of improvement include
performance, esp. on high-latency networks and the Internet, internationalization, interoperability with
non-POSIX operating systems, etc. Version 4.1 introduces Parallel NFS extension and other features designed to
support server clusters.

All the new features significantly increase complexity of the protocol -- for a rough measure, with over 600
pages, RFC~5661 is more than five times longer than the version 3 specification. This is obviously not
a problem in itself, but it does mean that understanding, implementing and maintaining correctness of NFSv4
software is a much more difficult task. Also, similar to SMB, the complexity makes it less suitable for
simpler file transfer tasks.

\section{SFTP}

SSH File Transfer Protocol, or SFTP, started out as a proprietary extension to the SSH~2.0~\cite{rfc4251}
protocol suite around the year 1997. The first standard draft was introduced in 2001, and standardization
attempts continued unsuccessfully for several years. Development was stopped and the working group closed
in~2006~\cite{secsh-email}. The most recent draft specificaton is from July 2006~\cite{draft-secsh-13}.
Despite the fact that SFTP is not a recognized Internet standard, it is widely implemented in SSH-related
software.

SFTP is a binary protocol with a clearly modern design. It has a small core feature set of filesystem-like
operations, but also includes support for extended attributes, ACLs, and byte-range locks.  The protocol
design allows for both stateful and stateless server implementations, but there is no form of replay
protection. Some consideration is given to cross-platform interoperability, but on the whole, the
functionality remains largely POSIX-centric.

Provisions for authentication and encryption are not part of the protocol, SFTP fully relies on the SSH
transport layer to provide these features.

% vim: tw=110:fo=an1lt:nosmartindent

\chapter{Design of the new protocol}

We have developed a new file transfer protocol called NewTP (as a play on FTP, the original file transfer
protocol). This chapter describes the design decisions that went into the development, how the protocol
operates and what features it provides. We also describe how the extension mechanism works and define several
extensions.

%%

\section{Overview}

NewTP is a binary application protocol running on an encrypted channel with ordered and reliable delivery. It
is designed with TCP + TLS in mind; for this version, use of TLS 1.2 or later is mandatory. Communication is
split into packets with a fixed-length header and at most 64 kB of data. Client initiates all traffic. For
every client request packet, server must send exactly one reply packet, and it must send the replies in the
order in which it received the requests.

After an initial handshake, client must authenticate itself using the SASL protocol~\cite{rfc4422}. Where
appropriate, anonymous authentication mechanism will be available. A conforming implementation must support at
least anonymous and password-based authentication.\footnote{That is, {\tt ANONYMOUS} and {\tt PLAIN} SASL
mechanisms.}

Once the session is established, client sends requests and receives replies in a common format. Every command
operates on an object refered to by a file handle. Before use, client must set up the file handle by assigning
a path string to it. Requests specify an arbitrary ID that is repeated in the reply; although this version of
the protocol does not allow it, future revisions might let the server reorder replies, in which case the
request ID will be used to match replies to corresponding requests.

Core NewTP command set consists of ranged read, ranged write with implicit file creation, truncate operation,
directory listing that includes selected metadata attributes, creating directories, delete and rename/move
operations, metadata querying and manipulation. Additional commands can be added through an extension
mechanism without changing the protocol.

%%

\section{Design decisions}

The protocol is intended as a lightweight file transfer protocol with some filesystem-like features. We wanted
to limit its feature set for ease of implementation, provide reasonable performance on high-latency networks,
ensure cross-platform interoperability, and make the protocol secure by default.

%

\subsection{Supported scenarios}

NewTP is designed with three main usecases in mind.

%

\subsubsection{Local fileserver}

In this scenario, the NewTP server is placed on a private local network or a VPN and provides a shared
document and multimedia store for its users. The network is considered a trusted environment, and the server
assumes that clients allowed to connect to the network are automatically entitled to access the document
store.  Encryption is not essential and authentication is completely unnecessary. Filesystem-like features are
desirable for operating system integration, but advanced locking and conflict resolution features are not
required, because conflicts will usually be resolved on the human level. This is the case for file servers or
media centers on home networks. Users in this scenario don't necessarily have technical background, so neither
the protocol nor its implementations should place any barriers to ease of use.

%

\subsubsection{Temporary share on a public network}

Users connected to the same network want to quickly exchange files. One of them sets up a NewTP server on
their computer, the others connect to it and download the published files or upload their own data. After all
transfers are done, the server is shut down. The network is assumed to be untrusted -- it might be an open WiFi
in a restaurant or a shared building-wide LAN -- so we require encryption, authentication and MitM protection.
However, the users are assumed to have a communication side-channel, such as close physical proximity, so
there is no need for any sort of PKI. The emphasis is on the ability to set up a server quickly, without
complicated supporting infrastructure. Ideally, both server and client software are implemented as portable
executables that the users can download and run with minimum of configuration. Filesystem-like features are
not required, because the users only need upload, download and directory manipulation capabilities.

%

\subsubsection{Shared Internet fileserver}

A web hosting or cloud storage provider sets up a NewTP server connected to the Internet. Clients of the
provider are issued credentials and use the NewTP protocol to access their files. Obviously, encryption and
MitM protection is required. However, in this scenario, it is more difficult to verify authenticity of the
server. Its certificate must either be verifiable through a PKI, or a through a side-channel such as physical
mail.  Implementations should also check for certificate changes.

A single server will be serving multiple users, so the protocol must provide user accounts and the
implementation will publish different parts of the filesystem based on which user account is authenticated.
Users will be using both dedicated NewTP clients and operating system integration, so filesystem-like
features, although not strictly required, are strongly desirable.

%

\subsection{Simplicity}

Simplicity means different things to different people and in different contexts. In case of NewTP, we want to
ensure that the protocol is simple to implement and use properly. The core feature set is small and
variability of commands is limited. Individual commands mostly correspond to how the operating system works
with filesystems, so each command can usually be implemented with one or two system calls on the server. On
the client, it is easy to replicate common file operations with calls over the network. In difficult cases, we
opt for less complex solutions that work well for our supported scenarios, even if they do not perfectly cover
corner cases.

Simplicity is also the reason for choosing TCP as the transport layer. This way, the protocol does not need to
concern itself with lost or out-of-order packets.

%

\subsection{Secure by default}

Protocols designed before the rise of Internet usually provide encryption as an optional feature. This leads
to widespread practice of using unencrypted connections on insecure public networks. We want to avoid this
practice with NewTP. Therefore, the protocol mandates use of TLS 1.2 or later for every connection.
Furthermore, we recommend that implementations use the highest available version of TLS at a given time, and
disable weak ciphersuites.

Especially in the ``local fileserver'' and ``temporary share'' scenarios, it is unrealistic to expect that
users would obtain a certificate from a trusted certificate authority, much less roll their own.
Implementations must allow self-signed certificates by default, and for ease of use, server applications
should generate their own certificates automatically. To ensure that attacker does not impersonate
a legitimate server with a brand new self-signed certificate, clients should remember certificates on first
connection and warn the user if the certificate has changed.\footnote{This practice is known as {\it
certificate pinning}.}

We attempted to design NewTP so that implementing it does not easily lead to buffer overflows.  Packet size is
limited to 64 kB, so every packet can fit into a fixed-size buffer. The protocol is binary and most packet
structures have a predefined size. Strings are not null-terminated, so the implementation is forced to use the
corresponding length field. Same is true for lists of structs. Where possible, length information is not
duplicated so implementations cannot be fooled by a mismatch. Simplicity of the protocol also means that the
client can't easily tie up disproportionately huge amounts of server processing power with a small number
of~requests.

%

\subsection{Low latency}

Given that the protocol is using TLS on top of TCP, establishing a connection takes a nontrivial amount of
time. This is even more true on high-latency networks like GPRS/EDGE. We want to make operation on such
networks as painless as possible. That means good support for interleaving of operations on a single
connection, and reducing size of protocol messages.

The choice of a binary protocol results in smaller messages. In addition, we recommend enabling TLS
compression. The short maximum packet length means that in the ideal case, after a new operation is inserted
into the outgoing stream, the client downloads at most 64 kB of data before receiving a response for the newly
inserted operation. In practice, clients on high-latency networks will batch requests, so that when the new
operation is inserted, it has to wait before the whole preceding batch is processed. However, there is room
for implementations to perform flow control and send batched messages only as fast as they can receive
responses, in which case it is possible to insert commands midway through.

This sort of interleaving rules out long operations, which is helpful even on fast networks. It means that it
is easy to implement a responsive filesystem emulation on top of NewTP.

Transfer overhead works out to 24 bytes per 64 kB when using maximum possible packet size. To put this in
perspective, when downloading 2 GB of data at top speed, protocol overhead will consume additional 786 kB.
This is perfectly acceptable.

%

\subsection{Authentication is optional}

In the ``local fileserver'' scenario, we don't want the clients to authenticate. In the ``temporary share'',
the authentication is supposed to be short and simple, probably based on a passphrase. In the ``Internet
fileserver'' scenario, strongest possible authentication is required, possibly including client certificates
or two-factor authentication.

We decided to implement authentication through the SASL protocol. This allows us to use a wide range of
authentication mechanisms, ranging from anonymous access, to plaintext usernames and password, to digest-based
authentication. Also, with SASL, new authentication mechanisms can be supported by implementations without
changing the NewTP protocol. Reusing an existing technology (SASL is implemented by a number of widely
available libraries) also reduces work for implementers.

The fact that SASL supports anonymous authentication allows us to make the SASL process mandatory, while still
technically keeping authentication optional. This again simplifies the protocol.

%

\subsection{Concurrency is out of scope}

NewTP is intended to service usecases with minimal interactions between users and negligible amounts of
conflicts. Because of that, we decided not to put any locking mechanisms into the protocol. Locking is
a decidedly complex problem in itself, and it is not a problem that NewTP is trying to solve.

We take concurrency issues into consideration and make reasonable attempts to prevent or solve them. We expect
that locking mechanisms could be implemented as protocol extensions, either by third parties or in a future
revision of the protocol. In general, however, we consider reliable concurrency to be out of scope, and
recommend using a different protocol for scenarios that require it.

%

\subsection{Caching policy}

In this version of the protocol, clients are fully responsible for managing their caches.  Depending on the
application, a client should periodically check freshness of relevant data.  For filesystem emulation, clients
should check cache freshness every time a file is opened, and perform server write when the file is closed.
In addition, it is advisable to refresh directory cache at reasonable intervals, e.g. once a minute.  (XXX
really?)

We considered adding a mechanism through which the server could request cache invalidation. However, since all
traffic is initiated by the client, server has no means to send unsolicited updates. Adding this would
complicate client implementations and we did not want that. Alternately, clients could register watched items
and then use a single command to poll their status periodically. We did not make this a part of the core
protocol, but a future revision might introduce an extension with this functionality.

%

\subsection{Non-idempotent operations}

Since native locking is not part of the protocol, we make an attempt to minimize problems with directory-based
locking and mitigate potential issues with non-idempotent operations. To this end, we have designed
a ``Replay'' extension. After establishing a session and authenticating, the client can ask for a unique
session token. If the connection is lost, the client can reconnect, and after a successful authentication, ask
for a replay of the session identified by that token. The server will reply with results for the last few
non-idempotent operations recorded under the provided token. If the session is still alive on the server, it
will be terminated to prevent late delivery of further commands. This way the client can resume after the last
command that was successfully received by the server.

Security of this extension is ensured by using tokens that are long enough to prevent guessing, and only
providing the replay to the same user account. Implementations might also check that the new connection is
coming from the same host as the old one.

%%

\section{Building blocks}

This section describes various technical details pertaining to design and implementation of NewTP.

%

\subsection{Packet format notation}

Packets are described in a C-style format, data type preceding field name. Various packets use data types
defined as follows:

\def\ttitem#1{ \item[\ttfamily #1] }
\begin{description}[leftmargin=1.7cm,style=sameline]
	\ttitem{byte} An 8-bit value (octet). Bytes represent command codes, result codes and other fields
		where the value is one of a fixed set of possibilities. The values are written out in
		hexadecimal notation, e.g. {\tt 0x07}.
	\ttitem{uint16} Unsigned 16-bit integer
	\ttitem{uint32} Unsigned 32-bit integer
	\ttitem{uint64} Unsigned 64-bit integer
	\ttitem{time\_t} Unsigned 64-bit integer, representing time in microseconds since the start of UNIX
		epoch (1 January 1970, 0:00:00 UTC)
	\ttitem{string} Byte sequence of arbitrary length, representing a Unicode string encoded in UTF-8.
		The field is not null-terminated. Length is not part of the {\tt string} field, a separate
		{\tt uint16} length information must precede it.
	\ttitem{data} Byte sequence of arbitrary length, representing arbitrary data. Same as {\tt string},
		length must be specified separately.
\end{description}

%

\subsection{Path string encoding}

All filenames and path strings, expressed as {\tt string} fields in packet descriptions, must be encoded as
UTF-8. Clients and servers must perform transcoding from system-native filename encoding to UTF-8 and back.

To support filenames that are undecodable in the system encoding, NewTP uses a roundtrip-safe encoding scheme
described in PEP~383~\cite{pep383}. In this scheme, an undecodable byte {\tt 0x80}--{\tt 0xFF} is converted to
a lone surrogate codepoint {\tt U+DC80}--{\tt U+DCFF}, which is then encoded to a 3-byte sequence by the UTF-8
algorithm. Upon receiving the string, reverse conversion or implementation-defined handling of the string is
performed.

Surrogates are codepoints reserved for UTF-16 encoding of codepoints above the Basic Multilingual Plane. They
are only valid as pairs in UTF-16, and not valid in UTF-8 at all. Therefore, a valid encoding cannot produce
a string which contains a lone surrogate codepoint, so there are no conflicts with this encoding scheme. Note
that this means that a string using this encoding scheme is not valid UTF-8. Implementations must be prepared
to handle potential UTF-8 decoding errors.

This encoding scheme does not cover all cases. Notably, in some encodings, bytes {\tt 0x00} through {\tt 0x7F}
can be parts of undecodable sequences. These are not supported for security reasons: it might be possible to
``smuggle in'' ASCII sequences such as `{\tt /..}', thus opening a new attack surface that must be checked by
implementations. See PEP~383~\cite{pep383} for full discussion. In NewTP, such bytes are instead to be
replaced by the Unicode replacement character {\tt U+FFFD}.

Also note that certain encodings, such as Big5, cannot be roundtrip-safely decoded to Unicode, because
multiple byte sequences map to the same Unicode codepoint. These are not supported by the core protocol.

%

\subsection{Path strings and file handles}

Paths and files in NewTP are referred to through file handles. The {\tt ASSIGN} command assigns a path string
to a file handle, and subsequent commands use the file handle as an argument, instead of the path string. The
purpose of this is to limit repeating of the potentially long path string.

Path strings are absolute and use the slash character (`{\tt /}') to separate path elements. Directory listing
returns names without full paths, and an absolute path name is constructed by combining directory path, slash, and
entry name from the listing. Paths do not start with leading slash -- a leading slash is considered to mean
that the first path element has zero length, which is invalid. Empty path string is allowed and refers to root
of the published filesystem.

File handle ID is specified as a 16-bit unsigned integer. That means that at most 65536 handles are available
to the client. Servers should provide as many handles as they can. At session initiation, server informs the
client about how many handles are available; these must be consecutive handles numbered 0--$({\tt max\_handles}
- 1)$. A conforming server must provide at least 16 handles.

File handles do not necessarily refer to valid filesystem objects -- for example, to create a new directory,
the client assigns the new path to a file handle and then issues a {\tt MAKEDIR} command on that handle.
A file handle does not become invalid when the object it is pointing to is removed, and it does not follow
externally moved files.\footnote{Although as a side-effect, the {\tt RENAME} command updates the file handle
with the new path.}

We wanted to avoid the necessity to open a file before writing, and the complexity associated with managing
``persistent'' open-file handles on the server. This means that several common POSIX usage patterns are not
directly supported: writing to a deleted or moved file recreates the original name instead of continuing to
write to the new location (or to the ``ghost'' version). These usage patterns can be emulated on the client if
necessary; replicating the POSIX semantics in concurrent scenarios is out of scope for NewTP.

%

\subsection{Metadata}

We define a set of basic attributes known about directory entries: entry type, size, creation time,
modification time, access rights etc. Each item is assigned a name, data type and a code. Recognized
attributes are listed in table~\ref{table:attributes}.

When requesting directory listing, the client specifies a sequence of codes, and specified metadata is
returned as part of each entry in the listing. Requested attributes from {\tt STAT} are specified in the same
way. Filesystem-like clients will request the full list, clients that want to save bandwidth can only ask for
attributes they are interested in.  Note that the full list is relatively long: three time fields and a size
field add up to 32 bytes, which is often longer than the entry name.

The {\tt SETATTR} call takes an attribute code and its new value. Only some attributes can be set, those are
marked as ``writable'' in the table. However, the fact that an attribute is writable does not mean that the
client has permission to set it.

Extended attributes and ACLs are not supported in the core protocol, but support can be added via extensions.
Extensions can also provide new attribute codes.

\begin{table}[p]
\begin{center}

\def\tline#1#2#3{
{\ttfamily #1} & {\ttfamily #2} & {\ttfamily #3}
}
\begin{tabular}{|lllcp{6cm}|}
	\hline
	Code & Name & Type & Writable & Value \\
	\hline
	\tline{0x00}{type}{byte} & No &
		Directory entry type: \newline
		{\tt 0x01} -- File \newline
		{\tt 0x02} -- Directory \newline
		{\tt 0x00} -- Other (e.g. device node) \newline
		{\tt 0xFF} -- Error entry
		\\
	\tline{0x01}{rights}{byte} & No &
		Bit mask of simplified access rights for~the current user account:\newline
		{\tt 0x01} -- Read permission \newline
		{\tt 0x02} -- Write permission \newline
		{\tt 0x03} -- Read and write permission
		\\
	\tline{0x03}{size}{uint64} & No & File size in bytes \\
	\tline{0x04}{device\_id}{uint32} & No &
		Arbitrary ID of logical device on which this entry resides.
		Entries within the same filesystem will have the same {\tt device\_id}
		and entries on different filesystmes will differ. No other guarantees
		are provided about the value.
		\\
	\tline{0x05}{links}{uint32} & No & Number of hard links to this entry \\
	\tline{0x06}{time\_access}{time\_t} & Yes & Last accessed time \\
	\tline{0x07}{time\_modified}{time\_t} & Yes & Last modification time \\

	\hline
	\multicolumn{5}{|c|}{\textit{POSIX-specific attributes}} \\
	\hline

	\tline{0x10}{posix\_mode}{uint32} & No &
		File type and permissions as expressed in the {\tt st\_mode} field
		of {\tt struct stat}.\newline
		TODO split type and permissions into separate fields?
		\\
	\tline{0x11}{time\_ctime}{time\_t} & Yes & POSIX {\tt ctime} \\
	\tline{0x12}{uid}{uint32} & Yes & User ID of owner \\
	\tline{0x13}{gid}{uint32} & Yes & Group ID of owner \\

	\hline
	\multicolumn{5}{|c|}{\textit{Windows-specific attributes}} \\
	\hline

	\tline{0x20}{attributes}{uint32} & Yes & Windows file attributes \\
	\tline{0x21}{time\_created}{time\_t} & Yes & Entry creation time \\

	\hline
\end{tabular}

\end{center}
\caption{Attribute codes, types and values}
\label{table:attributes}
\end{table}

%

\subsection{Extension mechanism}
\label{ssec:extensions}

\begin{samepage}
Each extension field is sent in the following format:
\begin{description}[parsep=1pt]
	\ttitem{byte extension\_code} \hfill \\
		8-bit extension code. Must not be zero.
	\ttitem{uint16 extension\_name\_len} \hfill \\
		Length of {\tt extension\_name}
	\ttitem{string extension\_name} \hfill \\
		Name of the extension
\end{description}
\end{samepage}

%%

\section{Communication scheme}

After establishing the TCP connection and successful TLS handshake, the client starts session initialization,
through which it agrees on a common version with the server. If the session was initialized successfully,
client starts the SASL authentication process. If that is successful, client can start sending commands.

%

\subsection{Session initalization}

\begin{samepage}
After the TLS session is established, client introduces itself with the following packet:
\begin{description}[parsep=1pt]
	\ttitem{string newtp} \hfill \\
		Fixed ASCII string ``{\tt NewTP}''
	\ttitem{uint16 length} \hfill \\
		Length of payload
	\ttitem{uint16 version} \hfill \\
		Protocol version
\end{description}

Here, ``payload'' refers to everything that comes after it. In this case, payload length is 2 (size of the
{\tt version} field). Protocol version is 1. We specify the payload length for extensibility: in a future
version, client might be sending more info in the initial packet. In version 1, only the version number is
sent.
\end{samepage}

\begin{samepage}
Server replies with the following packet:
\begin{description}[parsep=1pt]
	\ttitem{string newtp} \hfill \\
		Fixed ASCII string ``{\tt NewTP}''
	\ttitem{uint16 length} \hfill \\
		Length of payload
	\ttitem{uint16 version} \hfill \\
		Protocol version.
	\ttitem{uint16 max\_handles} \hfill \\
		Number of available file handles
	\ttitem{uint16 max\_opendirs} \hfill \\
		Number of directories available for simultaneous listing
	\ttitem{uint16 authstr\_len} \hfill \\
		Length of {\tt authstr}
	\ttitem{string authstr} \hfill \\
		List of supported SASL authentication mechanisms, separated by ASCII spaces ({\tt 0x20})
	\ttitem{uint16 num\_extensions} \hfill \\
		Number of extension fields TODO number or length?
	\ttitem{data extensions} \hfill \\
		Extension fields
\end{description}
\end{samepage}

Server should reply with a protocol version that is either the highest version supported, or equal to client
protocol version, whichever is lower. That means that if the server supports a version higher than the client,
it should downgrade to the client version. Server can only report higher version than the client if it is not
backwards-compatible with client version. In that case, client should probably abort the connection attempt.

If the client version is higher than the server version, client should either use the protocol version
supplied by server, or abort the connection attempt.

Format of extension fields is described in section \ref{ssec:extensions} Extension mechanism.

This concludes the session initialization. The client can now select an authentication mechanism and proceed
to the next step. If the server advertises the {\tt ANONYMOUS} mechanism, the client should select it, unless
the user indicates otherwise.

%

\subsection{SASL authentication}

This section of the work provides all relevant information in conformance with RFC~4422~\cite{rfc4422}
section~4 (Protocol Requirements). GSS-API service name is ``{\tt newtp}''.\footnote{The name is not
registered with IANA, so we do not technically conform to RFC~2743~\cite{rfc2743} requirements, and
transitively to RFC~4422 requirements either. For the purposes of this work, we consider this informal
conformance sufficient, seeing as hardly anyone will be using NewTP with GSS-API authentication in the
foreseeable future. At the time of this writing, the name ``{\tt newtp}'' is not allocated in the IANA service
registry.} NewTP does not allow multiple authentication, so any authentication exchange can be aborted by
disconnecting. This version of the protocol does not support separate authentication and authorization
identities, so authorization identity string should always be empty.

Mechanism negotiation is covered by the previous subsection. The core protocol does not allow the client to
rediscover available authentication mechanism after the authentication step is completed. We consider this
acceptable, because the session is already secured by TLS. For the same reason, the protocol is not using SASL
security layers even if they are available.

Scheme of the authentication exchange is described in RFC~4422. The client initiates the exchange, after which
the server can reply with a challenge and client reacts to that by sending a response. This can be repeated
zero or more times. At the end of the exchange, in reply to last client response, server sends an outcome
packet.  If the outcome is success, client is authenticated and can start sending commands. If the outcome is
failure, server will disconnect the client.

Client initiates the authentication exchange by sending one of the following packets:
\begin{enumerate}

\begin{samepage}
\item Initiation with mechanism name, {\it without} initial response
\begin{description}[parsep=1pt]
	\ttitem{string auth\_init} \hfill \\
		Fixed ASCII string ``{\tt A0}''
	\ttitem{uint16 length} \hfill \\
		Length of payload (in this case, only the mechanism name)
	\ttitem{string mechanism} \hfill \\
		Name of the selected mechanism
\end{description}
\end{samepage}

\begin{samepage}
\item Initiation with mechanism name, {\it with} initial response
\begin{description}[parsep=1pt]
	\ttitem{string auth\_init} \hfill \\
		Fixed ASCII string ``{\tt A1}''
	\ttitem{uint16 length} \hfill \\
		Length of payload
	\ttitem{uint16 mechanism\_len} \hfill \\
		Length of mechanism name
	\ttitem{string mechanism} \hfill \\
		Name of the selected mechanism
	\ttitem{uint16 response\_len} \hfill \\
		Length of initial response
	\ttitem{data response} \hfill \\
		Initial response
\end{description}
\end{samepage}

\end{enumerate}

\begin{samepage}
Server challenge packet format is as follows:
\begin{description}[parsep=1pt]
	\ttitem{string auth\_challenge} \hfill \\
		Fixed ASCII string ``{\tt AC}''
	\ttitem{uint16 length} \hfill \\
		Length of challenge
	\ttitem{data challenge} \hfill \\
		Contents of challenge
\end{description}
\end{samepage}

\begin{samepage}
Client response packet format is as follows:
\begin{description}[parsep=1pt]
	\ttitem{string auth\_response} \hfill \\
		Fixed ASCII string ``{\tt AR}''
	\ttitem{uint16 length} \hfill \\
		Length of response
	\ttitem{data response} \hfill \\
		Contents of response
\end{description}
\end{samepage}

At the end of the exchange, server will return an outcome in one of two ways:
\begin{enumerate}

\begin{samepage}
\item Outcome of authentication exchange, {\it without} additional data
\begin{description}[parsep=1pt]
	\ttitem{string auth\_outcome} \hfill \\
		Fixed ASCII string ``{\tt AF}'' (for ``Auth Finished'')
	\ttitem{uint16 length} \hfill \\
		Length of payload (in this case, 1)
	\ttitem{byte result} \hfill \\
		Result code
\end{description}
\end{samepage}

\begin{samepage}
\item Outcome {\it with} additional data
\begin{description}[parsep=1pt]
	\ttitem{string auth\_outcome} \hfill \\
		Fixed ASCII string ``{\tt AD}'' (for ``Auth outcome with Data'')
	\ttitem{uint16 length} \hfill \\
		Length of payload
	\ttitem{byte result} \hfill \\
		Result code
	\ttitem{uint16 adata\_len} \hfill \\
		Length of additional data
	\ttitem{data adata} \hfill \\
		Additional data
\end{description}
\end{samepage}

\end{enumerate}

TODO define codes for success and failure

%

\subsection{Request and response packets}

Request header contains command id, handle id and operation id, plus length of payload. Payload may contain
further arguments for the command. Response header contains payload length, operation id and return code.
Smaller than 0x80 is OK, 0x80 and above is error.  The codes have predefined meanings.

%

\subsection{Communication scheme}

Client starts by sending hello, then it authenticates, then it fires commands and gets responses.  In the
first version, responses come in the same order as commands. There is no QUIT command, client simply
disconnects and server discards all (most?) state. When establishing the session, different packet format is
used, this way a future version only needs to keep the hello message and redefine other packets.

%

\subsection{Example session}

TODO -- only a table?

%%

\section{List of commands}

Complete list of core commands, with descriptions, arguments, etc.

\subsection{ASSIGN}

\subsection{STAT}

\subsection{SETATTR}

\subsection{STATVFS}

\subsection{READ}

\subsection{WRITE}

\subsection{TRUNCATE}

\subsection{DELETE}

\subsection{RENAME}

\subsection{MAKEDIR}

\subsection{REWINDDIR}

\subsection{READDIR}


%%

\section{Result codes}

TODO -- only a table?

%%

\section{Extensions}

this section describes the concept of the extension mechanism and lists some extensions

\subsection{Asynchronous operations}

Commands to fire off an operation (equivalent of some commands, plus on-server copy, recursive delete,
cross-filesystem move etc.) on a handle, command to monitor progress (depends on the command, by default
"none/in progress/finished"), command to abort.

\subsection{Extended attributes}

Commands to read and set extended attributes.

\subsection{Links}

Separate extensions for hardlinks and softlinks? Each with one command: link this path to this new name.

\subsection{Replay}

Remember results of certain operations in the main process and replay them to whoever asks.

% vim: tw=110:fo=n1lt

\chapter{Implementation}

As a proof of the proposed concept, we have implemented a NewTP server and two client applications. This
served to evaluate soundness of the design ideas, test out protocol features, and prove that the specification
does not place unnecessary complications on the implementation. This chapter gives a brief overview of the
software, describes how it is implemented, and proposes directions for future improvements.

The provided implementation is purposefully simplistic. In many cases, we opted for straightforward but
inefficient approaches in place of more involved techniques that would improve performance or robustness.
Please keep in mind that this is purely a prototype and it is not intended for production use. The software is
architected with further development in mind, but it should be considered a mere foundation for a more
efficient and reliable implementation.

The prototype only implements the core command set, not the extensions.

The software that we have developed runs on the Linux operating system. It consists of three executable
programs built from a common code base. The {\tt server} program is the NewTP server and the {\tt newfs}
program implements the FUSE filesystem module. Lastly, {\tt client} is a simple command-line tool for
downloading individual files.

\section{Common functionality}

Certain operations are required both for client and server implementations. All three programs share the
headers that define command, result and attribute codes, and headers with definitions of {\tt struct}s that
correspond to NewTP packet formats and arguments. The shared functionality, apart from utility functions,
consists of the following:
\begin{itemize}[nolistsep]
	\item Packing and unpacking data, specified by a format string, to and from buffers
	\item Auto-generated helper functions that can pack and unpack the corresponding {\tt struct}s
		directly
	\item Helper functions to set up and tear down a TLS session
	\item Helper functions to send and receive data from the network
	\item Wrapper functions for these helpers that cleanly close the connection and exit the program
		when a network error occurs
\end{itemize}

\section{Server}

The server program allows the user to specify directories to export through NewTP as ``shares'', that is, virtual
entries in the published filesystem root. The user can choose which of them should be read-only and which are
writable. It is also possible to enable password-based authentication.

The server is implemented in a traditional forking model. The main process binds to ports on all available
interfaces and listens for incoming connections. When a client connects, the main process spawns a child
process to handle the incoming connection, and resumes listening.

The child process first establishes the TLS encryption, processes NewTP session initialization and SASL
authentication, and enters an endless command-serving loop. In each iteration, the process blocks on reading
from the network until it has enough data to decode an incoming request header. After that, it retrieves the
request payload, stores it in a buffer, which is then handed over to the appropriate command procedure. This
procedure decodes request arguments if required, performs the command and places its results into an output
buffer.  When the command procedure returns, contents of the output buffer are sent as a reply to the client.

Core commands are implemented in a separate source file and do not perform any network communication -- that
is the sole responsibility of the main server code. This enforces the notion that commands should not have
``private conversations'' with the client. The only data shared between command invocations is placed in file
handle structures. This is used to retain open directory descriptors between {\tt READDIR} operations, and
perform the basic optimization of keeping file descriptors open between read and write operations.

File handle assignment is performed by another separate code file, so that its functionality can be easily
exposed to extensions. This code is responsible for resolving the share-based virtual paths to actual
filesystem paths on the server. There are some provisions for the ability to dynamically add or remove shares
at run-time, but this functionality is not implemented in the prototype. At the moment, the coupling between
the file handle module and the core command module is still relatively tight. This should be rectified in
future versions.

Command procedures share the same basic structure, although it is more complicated esp. in cases where there
are more system calls in sequence ({\tt WRITE}) or more complicated processing is required on the result data
({\tt READDIR}). The general structure is as follows:
\begin{enumerate}[nolistsep]
	\item Obtain the file handle
	\item Perform a system call
	\item Resolve error code to the appropriate NewTP result code
	\item Store results in output buffer
\end{enumerate}

\section{Filesystem module client}

The primary client implementation of NewTP is a module for the Linux FUSE filesystem. It connects to a server
and exposes its contents through the Linux virtual file system interface, so that all programs running on
the system can access it transparently.

The FUSE filesystem was extremely helpful when developing the protocol, because it represents a convenient
user interface to individual filesystem operations. Once the basic {\tt STAT} and {\tt REWINDDIR}/{\tt
READDIR} operations were in place, we could use the UNIX command-line tools to test every feature of the
protocol separately. Furthermore, running more complicated actions within the filesystem (such as rebuilding
the executables from source) allowed us to stress-test the implementation and easily discover missing
functionality.

The FUSE library itself was also very useful in implementing the module. It performs a great deal of error
checking, replaces unimplemented operations with stubs, and ensures that arguments passed to the client
program are well-formed and within acceptable ranges. Combined with the common functionality, this lets us
implement most operations in 5 to 10 lines of C code, and keep everything in a single source file. In fact, by
far the most complicated part of the code is a primitive hash-table implementation for file handle lookup.

Every operation procedure receives a path string and arguments for that particular operation. With that, the
procedure performs the following steps:
\begin{enumerate}[nolistsep]
	\item Look up or assign a file handle for this path string
	\item Build an appropriate NewTP command with arguments
	\item Send it over the network
	\item Block until reply arrives
	\item Resolve NewTP result code to appropriate error code
	\item Return results of the operation
\end{enumerate}

FUSE modules can run in one of two modes: single-threaded, where operations are serialized, and
multi-threaded, where more than one operation can run at once. To properly use the multi-threading capability,
we would have to maintain a background service thread to handle network communication and marshal requests and
replies. This would needlessly complicate the proof-of-concept implementation. Therefore, the {\tt newfs}
program runs in single-threaded mode.

This is obviously a huge performance hit. Combined with the fact that there is no caching, it means that every
filesystem operation waits for a network round-trip before allowing a different filesystem operation to
proceed. When running over the local loopback interface, the response time is acceptable for day-to-day file
operations, but it is not suited for much more than that.

The obvious direction for future improvements is multi-threaded operation. A marshal thread will be in charge
of maintaining the connection to the server and recovering from network errors.  Individual operations will
make blocking calls into this thread, it will send requests on their behalf and unblock the caller when the
corresponding reply arrives. The marshal thread would also be in a perfect position to manage application-wide
caches of files and directory entries, perform read-ahead requests, and respond to server notifications.

\section{Downloader client}

The {\tt client} program's main purpose is downloading files faster than {\tt newfs}. It was developed as
a scratch pad for implementation practices. Apart from downloading, it can also list directory contents
in {\tt ls}-like format. If developed further, it could become a general-purpose tool for working with NewTP
in shell scripts.

The implementation is very straightforward: the main procedure establishes a connection to the server, and
launches the appropriate procedure based on command line arguments. These procedures don't follow any
particular pattern. Each of them performs its task sequentially, prints results if appropriate, and exits the
program.

As stated above, the filesystem client waits for results of every read operation before issuing a new one. We
attempted to improve the performance of {\tt client} by sending read operations in batches. At start of the
session, it sends eight consecutive read requests before it starts waiting for replies. Every time a reply
arrives, the program issues a new read request. This way, there should be outstanding requests and replies at
all times, so the network bandwidth is better utilized.

In informal testing, the batch approach increased performance over the naïve approach by a factor between 50
and 100. This proves that the concept has merit. However, when compared with a mature HTTP server,
our NewTP implementations still leave a lot to be desired.


% vim: tw=110:fo=an1lt

\chapter*{Conclusion}
\addcontentsline{toc}{chapter}{Conclusion}

We have successfully designed a file sharing protocol with filesystem-like capabilities. The protocol fulfils
some of its stated goals: it is easy to implement, communicates securely, and allows for operation
interleaving.  It also has strong provisions for extensibility and cross-platform interoperability.  Further
research is required to determine if the protocol exhibits the expected characteristics on unreliable and
high-latency networks.

We have proven that the design is viable by implementing and testing a server and a virtual filesystem backed
by the proposed protocol. The prototype allowed us to show that the protocol has all the features needed to
reliably function as a filesystem. This make it possible to integrate it into an operating system.

In our future work, we would like to focus on improving performance of the implementations, to the point where
they are usable for practical purposes, and then observe how they perform in real-world scenarios. We have
briefly outlined some possible performance improvements in Chapter~4.

There is also room for improvement in the protocol specification. Chapter~3 mentions a number of
possible future advancements, including reordering of replies, notifications from the server, and a locking
extension.


% Ukázka použití některých konstrukcí LateXu (odkomentujte, chcete-li)
%\include{example}

%%% Seznam použité literatury
%%% Seznam použité literatury je zpracován podle platných standardů. Povinnou citační
%%% normou pro bakalářskou práci je ISO 690. Jména časopisů lze uvádět zkráceně, ale jen
%%% v kodifikované podobě. Všechny použité zdroje a prameny musí být řádně citovány.

\def\bibname{Bibliography}
\begin{thebibliography}{99}
\addcontentsline{toc}{chapter}{\bibname}

\bibitem{lamport94}
  {\sc Lamport,} Leslie.
  \emph{\LaTeX: A Document Preparation System}.
  2. vydání.
  Massachusetts: Addison Wesley, 1994.
  ISBN 0-201-52983-1.

% Chapter 1

\bibitem{asn-basic}
	{\sc International Telecommunication Union}.
	ITU-T Recommendation X.680:
	\emph{Information technology -- Abstract Syntax Notation One (ASN.1): Specification of basic notation}.
	Geneva, Switzerland: ITU-T, 2008.
	ITU-T Recommendation series X, X.680.
	Available from: \url{http://www.itu.int/rec/T-REC-X.680-200811-I}

\bibitem{rfc5246}
	{\sc Dierks,} T. and E. {\sc Rescorla}.
	\emph{The Transport Layer Security (TLS) Protocol Version 1.2}
	[online].
	RFC 5246.
	Internet Society, 2008 [viewed 4 December 2013].
	Available from: \url{http://tools.ietf.org/html/rfc5246}

% Chapter 2

\bibitem{rfc454}
	{\sc McKenzie,} A.
	\emph{File Transfer Protocol -- meeting announcement and a new proposed document}
	[online].
	RFC 454.
	Internet Society, 1973 [viewed 4 December 2013].
	Available from: \url{http://tools.ietf.org/html/rfc454}

\bibitem{rfc959}
	{\sc Postel,} J. and J. {\sc Reynolds}.
	\emph{File Transfer Protocol}
	[online].
	RFC 959.
	Internet Society, 1985 [viewed 4 December 2013].
	Available from: \url{http://tools.ietf.org/html/rfc959}

\bibitem{rfc4217}
	{\sc Ford-Hutchinson,} P.
	\emph{Securing FTP with TLS}
	[online].
	RFC 4217.
	Internet Society, 2005 [viewed 4 December 2013].
	Available from: \url{http://tools.ietf.org/html/rfc4217}

\bibitem{rfc2616}
	{\sc Fielding,} R. et al.
	\emph{Hypertext Transfer Protocol -- HTTP/1.1}
	[online].
	RFC 2616.
	Internet Society, 1999 [viewed 4 December 2013].
	Available from: \url{http://tools.ietf.org/html/rfc2616}

\bibitem{rfc2069}
	{\sc Franks,} J. et al.
	\emph{An Extension to HTTP : Digest Access Authentication}
	[online].
	RFC 2069.
	Internet Society, 1997 [viewed 4 December 2013].
	Available from: \url{http://tools.ietf.org/html/rfc2069}

\bibitem{rfc2818}
	{\sc Rescorla,} E.
	\emph{HTTP Over TLS}
	[online].
	RFC 2818.
	Internet Society, 2000 [viewed 4 December 2013].
	Available from: \url{http://tools.ietf.org/html/rfc2818}

\bibitem{rfc2518}
	{\sc Goland,} Y. et al.
	\emph{HTTP Extensions for Distributed Authoring -- WEBDAV}
	[online].
	RFC 2518.
	Internet Society 1999 [viewed 4 December 2013].
	Available from: \url{http://tools.ietf.org/html/rfc2518}

\bibitem{rfc4918}
	{\sc Internet Engineering Task Force}.
	RFC 4918:
	\emph{HTTP Extensions for Web Distributed Authoring and Versioning (WebDAV)}
	[online].
	Edited by L. Dusseault.
	June 2007 [viewed 4 December 2013].
	Available from: \url{http://tools.ietf.org/html/rfc4918}

\bibitem{rfc3253}
	{\sc Clemm,} G. et al.
	\emph{Versioning Extensions to WebDAV (Web Distributed Authoring and Versioning)}
	[online].
	RFC 3253.
	Internet Society, 2002 [viewed 4 December 2013].
	Available from: \url{http://tools.ietf.org/html/rfc3253}

\bibitem{mssmb2}
	{\sc Microsoft Corporation}.
	{\emph Server Message Block (SMB) Protocol Versions 2 and 3}
	[online].
	MS-SMB2.
	Revision 44.0.
	Microsoft Corporation, 25 October 2013 [viewed 4 December 2013]

% Chapter 3

\bibitem{rfc4422}
	{\sc Internet Engineering Task Force}.
	RFC 4422:
	\emph{Simple Authentication and Security Layer (SASL)}
	[online].
	Edited by A. Melnikov and K. Zeilenga.
	June 2006 [viewed 4 December 2013].
	Available from: \url{http://tools.ietf.org/html/rfc4422}

\bibitem{kelsey2002}
	{\sc Kelsey,} John.
	Compression and Information Leakage of Plaintext.
	In: \emph{Fast Software Encryption, 9th International Workshop, FSE 2002,
	Leuven, Belgium, February 4-6, 2002, Revised Papers}
	Berlin: Springer Berlin Heidelberg, 2002, pp. 263--276.
	ISBN 978-3-540-45661-2.
	Available from: \url{http://www.iacr.org/cryptodb/archive/2002/FSE/3091/3091.pdf}

\bibitem{pep383}
	{\sc v. Löwis,} Martin.
	\emph{Non-decodable Bytes in System Character Interfaces}
	[online].
	PEP 383.
	Python Software Foundation, 2012 [viewed 4 December 2013].
	Available from: \url{http://www.python.org/dev/peps/pep-0383/}

\bibitem{rfc2743}
	{\sc Linn,} J.
	\emph{Generic Security Service Application Program Interface Version 2, Update 1}
	[online].
	RFC 2743.
	January 2000 [viewed 4 December 2013].
	Available from: \url{http://tools.ietf.org/html/rfc2743}

\bibitem{xwindow}
	{\sc Scheifler,} Robert W. and James {\sc Gettys}.
	\emph{X Window System: Core and extension protocols: X version 11, releases 6 and 6.1}.
	Edited by Al Mento and Donna Converse.
	Boston: Digital Press, 1997.
	ISBN 978-1-55558-148-0

\bibitem{rfcXX}
	{\sc Internet Engineering Task Force}.
	RFC XX:
	\emph{}
	[online].
	Edited by .
	 [viewed 4 December 2013].
	Available from: \url{http://tools.ietf.org/html/rfcXX}


\end{thebibliography}


%%% Přílohy k bakalářské práci, existují-li (různé dodatky jako výpisy programů,
%%% diagramy apod.). Každá příloha musí být alespoň jednou odkazována z vlastního
%%% textu práce. Přílohy se číslují.
\chapwithtoc{Attachments}
\appendix
% vim: tw=110:fo=n1lt

\chapter{CD-ROM contents}

\begin{description}[nolistsep,leftmargin=2cm]
\item[{\tt 2013-jan-matejek-bc-thesis.pdf}] \hfill \\
	The text of this work in PDF format
\item[{\tt newtp-0.1.tar.bz2}] \hfill \\
	Source code archive (tarball) of the prototype implementation of NewTP
\item[{\tt newtp-0.1/}] \hfill \\
	Source code of the prototype implementation of NewTP. \\
	This directory contains the extracted tarball for convenient access
\item[{\tt bin-x86\_64/}] \hfill \\
	Compiled binaries of the prototype implementation. \\
	Dynamically linked binaries for 64-bit Linux:
\item[{\tt bin-x86\_64/server}] \hfill \\
	Server binary
\item[{\tt bin-x86\_64/newfs}] \hfill \\
	FUSE filesystem module binary
\item[{\tt bin-x86\_64/client}] \hfill \\
	Downloader client binary

\end{description}

\chapter{README for the software}
\begin{verbatim}
NewTP version 0.1
=================

This is a short guide to compiling and using the NewTP
prototype implementation.

1. Requirements
---------------

The prototype is designed to run on the Linux operating
system. Compatibility with other operating systems is not
guaranteed.

To compile and run the prototype, you must have the
following software:

 * a C compiler
 * GNU-compatible Make program
 * Python 3
 * pkg-config
 * GNU SASL Library (gsasl) version 1.6 or higher
 * GnuTLS version 3.0 or higher
 * FUSE version 2.6 or higher

2. Compiling
------------

Follow these steps to compile the software:

1. Extract the source tarball
   $ tar xvjf newtp-0.1.tar.bz2
2. Go into the directory
   $ cd newtp-0.1
3. Start build process
   $ make

3. Running
----------

3.1 server
----------

usage: ./server [-p password] <shares>

If the -p argument is not given, server runs in anonymous mode.
Shares can be specified as follows:

 /path/to/share=name - this share is read-only
 -ro /path/to/share=name - this is also read-only
 -rw /path/to/share=name - this is read-write


3.2 client
----------

usage: ./client <hostname> <command> [path]

client can be invoked in one of three modes:

 - list contents of root directory:
   $ ./client <hostname> list
 - list contents of path:
   $ ./client <hostname> list /path/of/interest
 - download a remote file:
   $ ./client <hostname> get /path/to/fi.le


3.3 newfs
---------

usage: ./newfs <hostname> <mountpoint>

Mounts the filesystem exported on <hostname> onto the
directory <mountpoint>. To unmount, use the command:

fusermount -u <mountpoint>


\end{verbatim}


\openright
\end{document}
