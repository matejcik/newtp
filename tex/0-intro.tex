% vim: tw=110:fo=an1lt

\chapter{Problem Definition}

In this chapter, we present an overview of the problem area and a description of what exactly we are solving.
It should also serve as an introduction for readers unfamiliar with the topic, providing explanations of terms
that we are using throughout this work.


\section{File Transfer}

Most modern operating systems share a common idea of what constitutes a filesystem. It is a collection of {\it
files} that are referenced through a hierarchical tree-like structure of {\it directories}. You can identify
any file or directory by its {\it path}, which is a string of entry names delimited by path separator
characters\footnotemark[1]. You follow the (absolute) path by starting at the root directory and always
descending into the directory named by the next path element, until the last element is remaining, naming the
target entry.

We want to be able to publish one or more subtrees of the filesystem on one computer, and access this part of
the structure from another computer over a network.


\section{Protocols}


\subsection{Sessions}

The most basic "protocol" imaginable is one computer just firing data packets at another, silently hoping that
the other will reassemble and understand them. We could, in theory, use this sort of protocol for transferring
files, but it is obviously very impractical. All but the most basic protocols will first establish a session,
in order to have a context for further communication. Extent of the session can vary, from simply marking
a collection of packets as belonging together, to providing ordered delivery, to authenticating parties and
encrypting transmitted data.

Basic session functionality is usually provided by the transport layer, specifically the Transmission Control
Protocol (TCP). This allows the communicating parties to establish a two-way stream of octets, with reliable
and ordered delivery, and flow control. Unless noted otherwise, protocols described in this work are built on
top of TCP.


\subsection{Encoding}

Protocols consist of messages exchanged by the communicating parties. There are two main ways to encode the
messages: either as human-readable text, or as custom binary data.

\subsubsection{Text-based protocols}

In text-based protocols, each message is a string of words and/or numbers, nowadays usually encoded as ASCII
or UTF-8. Many such protocols also use line separator characters to delimit messages. The main advantage of
this approach is that it is readable (and writable) by humans without specialized debugging tools - for
instance, it is possible to communicate with the other party "by hand", without software implementing the
protocol. Historically, there was also the fact that two communicating systems only had to agree on text
encoding, and not about in-memory format of numbers or other data structures.

Drawbacks include overhead of text encoding, both in terms of size and processing time spent encoding and
decoding the textual representation. Another problem is that lengths of message fields are not fixed, so
parsing and framing messages is more difficult.

\subsubsection{Binary protocols}

The alternative to text-based messages is using a format similar to in-memory encoding of data. Each message
can be described by, and parsed as, a C struct\footnotemark[2]. This results in shorter messages and simpler
encoding. If the in-memory encoding is the same as in the protocol, encoding and decoding can even be skipped
completely. Parsing and framing is trivial - every kind of message can either have a fixed number of
fixed-size fields, or it can have a fixed-size part describing lengths of all variable-size fields.

Even though today's worldwide computing environment is homogenous in many regards, care must be taken when
communicating between systems that differ in endianness. The protocol either has to prescribe the byte order,
or it must provide means for the parties to negotiate.

Unlike text-based protocols, debugging is not straightforward, humans require specialized tools that convert
the binary messages to understandable text.


\subsection{Roles}

We recognize two basic roles for communicating parties: {\it client} and {\it server}. Client is the party
that sends requests - specifies what is to be done - and server's task is to fulfil the requests and possibly
report on results.

Most protocols are modelled as client-server, with roles of the parties fixed in advance. Typically, the
server is listening for incoming connections and clients connect at their convenience. In a peer-to-peer
model, after establishing the connection, parties can act as both clients and servers at the same time,
sending requests to each other.

In this work, we use the following convention: a {\it server} will be the computer that contains a "master
copy" of a filesystem, and publishes a subtree of this filesystem on the network. A {\it client} is the
computer that wants to manipulate this published tree using the protocol. 


%%%

\footnotetext[1]{On UNIX-like systems, the path separator is a forward-slash, '/'. Windows recognizes both
forward-slash and a backslash, but backslash is used more commonly. Nevertheless, we will be using
forward-slash as a path separator everywhere.}

\footnotetext[2]{This does not mean that you can create a message by dumping the memory representation of
a corresponding C struct directly. On modern systems, fields of a struct will be memory-aligned and will
contain some amount of padding. This can differ between systems, compilers, and can even depend on particular
compiler flags.}
