% vim: set tw=110 fo=an1lt

\chapter{Problem Definition}

In this chapter, we present an overview of the problem area and a description of what exactly we are solving.
It should also serve as an introduction for readers unfamiliar with the topic, providing explanations of terms
that we are using throughout this work.

\section{File Transfer}

Most modern operating systems share a common idea of what constitutes a filesystem. It is a collection of {\it
files} that are referenced through a hierarchical tree-like structure of {\it directories}. You can identify
any file or directory by its {\it path}, which is a string of entry names delimited by path separator
characters\footnotemark[1].  You follow the path by starting at the root directory\footnotemark[2] and always
descending into the diretory named by the next path element, until the last element is remaining, naming the
target entry.

We want to be able to publish one or more subtrees of the filesystem on one computer, and access this part of
the structure from another computer over a network.

\section{Protocols}

The most basic "protocol" imaginable is one computer just firing data packets at another, silently hoping that
the other will reassemble and understand them. We could, in theory, use this sort of protocol for transferring
files, but it is obviously very impractical.



\footnotetext[1]{On UNIX-like systems, the path separator is a forward-slash, '/'. Windows recognizes both
forward-slash and a backslash, but backslash is used more commonly. Nevertheless, we will be using
forward-slash as a path separator everywhere.}

\footnotetext[2]{This is a so-called {\it absolute} path, usually denoted by a leading path separator.}
