% vim: tw=110:fo=an1lt:nosmartindent

\chapter{Problem Definition}

In this chapter, we present an overview of the problem area and a description of what exactly we are solving.
It should also serve as an introduction for readers unfamiliar with the topic, providing explanations of terms
that we are using throughout this work.


\section{Files and filesystems}

Most modern operating systems share a common idea of what constitutes a filesystem. It is a collection of {\it
files}, which are opaque strings of binary data. Individual files are referenced through a hierarchical
tree-like structure of {\it directories}.  You can identify any file or directory by its {\it path}, which is
a string of entry names delimited by path separator characters.\footnote{On UNIX-like systems, the path
separator is a forward-slash, `\texttt{/}'. Windows recognizes both forward-slash and a backslash, but
backslash is used more commonly. Nevertheless, we will be using forward-slash as a path separator everywhere.}
You follow the (absolute) path by starting at the root directory and always descending into the directory
named by the next path element, until the last element is remaining, naming the target entry.

We want to be able to publish one or more subtrees of the filesystem on one computer, and access this part of
the structure from another computer over a network.


\section{Protocols}

A protocol, simply put, is a communication script for two or more parties. It describes the conversation
structure, semantics and syntax of messages. Here is an overview of several issues that must be solved by
every protocol.


\subsection{Sessions}

The most basic ``protocol'' imaginable is one computer just firing data packets at another, silently hoping
that the other will reassemble and understand them. It is, of course, possible to use this sort of protocol
for transfering files, but it is obviously very impractical. All but the simplest of protocols will first
establish a session, in order to have a context for further communication. Extent of the session can vary,
from simply marking a collection of packets as belonging together, to providing ordered delivery, to
authenticating parties and encrypting transmitted data.

Basic session functionality is often provided by the transport layer, specifically the Transmission Control
Protocol (TCP). This allows the communicating parties to establish a two-way stream of octets, with
reliability, ordered delivery and flow control. Unless noted otherwise, protocols described in this work are
built on top of TCP.


\subsection{Encoding}

Protocols consist of messages, which must be encoded for over-the-wire transfer. There are two main ways to do
this: either as human-readable text, or as custom binary data.

\subsubsection{Text-based protocols}

In text-based protocols, each message is a string of words and/or numbers, nowadays usually encoded as ASCII
or UTF-8. Many such protocols also use line separator characters to delimit messages. The main advantage of
this approach is that it is readable (and writable) by humans without specialized debugging tools - for
instance, it is possible to communicate with the other party by typing out commands, without software
implementing the protocol.  Historically, there was also the fact that two communicating systems only had to
agree on text encoding, and not about in-memory format of numbers or other data structures.

Drawbacks include overhead of text encoding, both in terms of size and processing time spent encoding and
decoding the textual representation. Another problem is that lengths of message fields are not fixed, so
parsing and framing messages is more difficult.

\subsubsection{Binary protocols}

The alternative to text-based messages is using a format similar to in-memory encoding of data. Each message
can be described by, and parsed as, a C struct.\footnote{This does not mean that you can create a message by
dumping the memory representation of a corresponding C struct directly. On modern systems, fields of a struct
will be memory-aligned and will contain some amount of padding. This can differ between systems, compilers,
and can even depend on particular compiler flags.} This results in shorter messages and simpler encoding. If
the in-memory encoding is the same as in the protocol, encoding and decoding can even be skipped completely.
Parsing and framing is trivial - every kind of message can either have a fixed number of fixed-size fields, or
it can have a fixed-size part describing lengths of all variable-size fields.

Care must be taken when communicating between systems that differ in endianness.  The protocol either has to
prescribe the byte order, or it must provide means for the parties to negotiate.

Unlike text-based protocols, debugging is not straightforward, humans require specialized tools that convert
the binary messages to understandable text.


\subsection{Roles}

We recognize two basic roles for communicating parties: {\it client} and {\it server}. Client is the party
that sends requests - specifies what is to be done - and server's task is to fulfil the requests and report on
results.

Most protocols are modelled as client-server, with roles of the parties fixed in advance. Typically, the
server is listening for incoming connections and clients connect at their convenience. In a peer-to-peer
model, after establishing the connection, parties can act as both clients and servers at the same time,
sending requests to each other.

In this work, we use the following convention: a {\it server} will be the computer that contains a ``master
copy'' of a filesystem, and publishes a subtree of this filesystem on the network. A {\it client} is the
computer that wants to manipulate this published tree using the protocol.


%%%

\section{Capabilities of a file transfer protocol}

\subsection{Paths and filenames}

In order to accomplish anything useful with a protocol, there must be a way to specify on which file you want
to operate. This alone poses a number of issues that must be resolved.

\subsubsection{Relative paths}

The trivial way to represent a path is by using an absolute path string, i.e. one that specifies all path
components leading to the target object. Apart from this method, operating systems have a concept of {\it
current directory} and paths relative to the current directory. Relative path can contain entries named
``\texttt{.}'' (refers to current directory) and ``\texttt{..}'' (parent directory).\footnote{For security
reasons, protocol should prescribe, and implementations enforce, that ``\texttt{.}'' and ``\texttt{..}''
components are forbidden in an absolute path.} In some cases, it can be useful to expose this functionality
through the protocol - either allowing the client to choose the current directory, or setting it at a fixed
path (e.g.  authenticated user's home directory).

\subsubsection{Validity of path string}

Different platforms have different conventions for filenames and paths. Some characters might be invalid in
file and path names, lengths of path components or the total length of path might be limited. Protocols should
either abstract away the differences (an inherently difficult problem, usually resulting in only the
lowest-common-denominator functionality), provide means to query platform limitations, or at least acknowledge
failure modes for path names and note that they are platform-dependent.

Another problem is correspondence between directory listings and results of operations. On Windows, filenames
are not case-sensitive, so creating file {\tt foo} will fail if the directory already contains file {\tt FOO}.
Mac~OS~X normalizes unicode representations of filenames, which can mean that a newly created file will appear
under a different name.

\subsubsection{Filename character encoding}

One place where a file transfer protocol cannot avoid character encoding issues is in path specifiers and
directory entry names. The problem is that different systems have different conventions for filename encoding.
A protocol could specify that filenames are transfered as ``plain data'', but this would only push the problem
down to individual implementers. A better solution is to either prescribe or negotiate a transfer encoding,
and then transcode filenames in client and server software.

Newer filesystems enforce that a filename is a valid Unicode string, but on many widely used filesystems,
filename can be an arbitrary string of bytes.\footnote{Unless the byte represents a forbidden character, e.g.
forward-slash or null byte on UNIX, shell expansion characters on Windows etc.} In effect, a filesystem can
store names that cannot be properly decoded in the system's character set. This presents a problem for the
transcoding process.  An obvious solution is to ignore invalid filenames, but that is arguably not very
useful. It is seemingly better to use replacement characters for the undecodable sequences, but in practice,
this does not help very much, as this way of encoding is not roundtrip-safe. The client can get a listing that
contains invalid filenames, but cannot specify them, because files with the modified names do not actually
exist on the server. Moreover, it is impossible for the client to tell apart invalid filenames from valid ones
that happen to contain the replacement characters, unless the replacement characters are forbidden in valid
names.

Explicit support for undecodable filenames is possible in several ways, e.g. through lossless transcoding of
undecodable characters, or by refering to files in some other way than by name.


\subsection{Download}

Downloading, i.e. transfering files from server to client, is arguably the most basic capability provided by
a file transfer protocol. Client will request a file and server will respond by sending a binary stream of the
file's contents. Failure modes are obvious: the identifier does not point to a file; file exists but cannot be
read; or this client does not have permission to access the file.

In basic form of download, the server will send full contents of the file as a continuous stream, possibly
preceded by length of the stream. However, it is often more useful to specify a range for reading. This allows
resuming of interrupted downloads, seeking in media streams and the like. It is also very useful if the client
wants to implement filesystem-like behavior over the protocol.

A general-purpose protocol will treat files as opaque streams. However, in a specialized protocol, it might be
useful to understand structure of the file. This can allow the client to e.g. query contents of the file as
a database, download only thumbnail or embedded metadata etc.

\subsection{Upload}

Transfering from client to server is more complicated. In addition to checking for permissions, the server
must ensure that there is enough disk space, resolve issues with overwriting existing files, as well as
gracefully handle concurrent modifications. There might also be issues with missing path components - trying
to create file in a directory that does not exist. This results in several new failure modes that must be
expressed in the protocol. However, at this level, it is sufficient to add new error codes, or specify that
a single error code covers many different types of failure.

\subsection{Directories}

Although it might be tempting to handle directories as a special case of files, directories are very
different. Where files are opaque blobs, directories are inherently structured, so the protocol must represent
an API for manipulating the structure. A general-purpose protocol should provide at least the following
operations:

\begin{itemize}
	\item list contents of directory
	\item create file/directory
	\item delete file/directory
	\item rename a directory entry
\end{itemize}

Listing is pretty straightforward. The only potential issue is that it is difficult to know its length in
advance. This makes certain kinds of message framing more challenging.

Files can be created implicitly when attempting to write to them. Creating an empty file can be accomplished
by a zero-length write. On the other hand, there is no equivalent of ``zero-length write'' for implicit
creation of directories. Therefore, creating a directory needs to be a separate operation.

Deleting files and directories is similar in that it means removing one directory entry. However, if
implemented in a same operation, it has to account for the possibility that the directory being removed is not
empty. Then it can either offer the option to recurse into non-empty directories, or return a special error
condition to indicate this.

\subsection{Renaming and moving}

Within a single filesystem, renaming and moving is usually performed by the same operation, which is
essentialy ``change this complete path to a different one''. Since a move operation only consists of relinking
a directory entry, it functions the same on individual files as it does on directories, even non-empty
ones.\footnote{Unlike the remove operation, rename/move does not affect contents of the directory.} A protocol
can export this functionality as a single command.

Things get more difficult when the new pathname would end up on a different filesystem. To move a file from
one filesystem to another, you first need to make a copy of the file, then remove the original.  This is hard
to do atomically, even in case of a single file. When attempting to move a non-empty directory, the
copy-and-remove operation has to be done recursively on its contents.

In effect, rename/move operations either need to be (potentially) non-atomic, or fail on operations crossing
filesystem boundaries. In the latter case, the client would be expected to perform the move ``by hand'', by
copying to target and removing sources. But to do this efficiently, the protocol would have to expose an
on-server copy operation, otherwise the client would need to download and reupload all of the data.

\subsection{Server filesystem structure}

There are two reasons for the client to be interested in the server's filesystem configuration: free space
information, and knowledge about filesystem boundaries. Exposing this information is particularly complicated
on UNIX-like systems, where it is common to use arbitrary directories as mount points for different
filesystems. One possible solution is having a "statvfs"\cite{statvfs} operation that returns filesystem
information for particular file or directory. It would then be the client's responsibility to use this
operation whenever necessary. Another possibility is to make filesystem information part of directory listing.

\subsection{Metadata}

Apart from contents of files and directories, the filesystem stores metadata about entries: file size,
creation date, last modification date, filesystem attributes, ownership and access rights, link count etc.
Although it is not essential, a full-featured protocol will expose operations to query and modify this kind of
metadata. It is also very useful to send this information along with directory listing - querying each entry
in a directory with hudreds or thousands of files would be woefully inefficient. On the other hand, it might
be useful to specify what kind of metadata to include in the listing.

Filesystem metadata is obviously different on different filesystems and different operating systems. The
protocol can provide a common subset, a union set with missing fields, or platform-specific functions.

Some filesystems support custom metadata fields, or access control lists (ACLs) for fine-grained rights
management. These are rarely used in end-user software, but if a protocol is supposed to provide
filesystem-like functionality, it should also expose this kind of metadata.

\subsection{Locking}

Modern operating systems provide mechanisms for one process to lock a file, preventing other processes from
accessing or modifying it concurrently. It is straightforward to expose these operations through a network
protocol, but doing so without careful consideration will lead to problems. Locking is usually designed and
used as a guarantee of atomicity and/or consistency, and it is difficult to provide such guarantees when
working over an unreliable network. Transmission errors and lost connections may result in an inconsistent
state of the locked file if the lock is released too soon, or in deadlocks if the lock is held for too long.

%%%

\section{Other considerations}

When designing a protocol, it is not enough to look at a checklist of desired capabilities. We must also
consider how the protocol behaves in practice, and how it functions as a part of a larger system. In this
section, we examine various properties of the protocol and its behaviors in a more goal-oriented manner.

\subsection{Authentication}

Apart from file serving capabilities, a protocol should have means to identify clients and verify that they
are who they claim to be in order to properly enforce access rights.  Some protocols include authentication
mechanisms directly, others rely on their transport layer to provide
