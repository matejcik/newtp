\chapter{Related works}

This chapter examines several popular existing protocols and general characteristics
of their implementations.

\section{FTP}

FTP, the File Transfer Protocol, is one of the oldest network protocols for file transfer, and probably the
oldest that is still widely used. It was first introduced in~1973 as RFC~454~\cite{rfc454} and the most recent
accepted standard is RFC~959~\cite{rfc959} from~1985. Several proposed extensions were published, mostly
describing how the standard has evolved in practical use.

With FTP, the client establishes a text-based, line-oriented control connection to the server. Through this
the user can configure session parameters, traverse and list directories and manipulate directory entries,
similar to an interactive shell. Many commands are designed for human use, e.g. the HELP command, or the LIST
command that lists contents of the current directory in a platform-dependent textual format. Actual file
transfers run on separate data connections which are negotiated on the control connection. This means that in
order to support FTP transparently, firewalls and NATs must understand the FTP protocol.

Users authenticate themselves with user name, password and optionally an account name. The protocol specifies
no encryption, so this information is transferred in plain text. RFC~4217~\cite{rfc4217} specifies a variant
of FTP that uses TLS for encryption

\section{WebDAV}

\section{SMB/CIFS}

\section{NFS}

\section{SFTP --- SSH File Transfer Protocol}

\section{Others}
