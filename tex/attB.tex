\chapter{README for the software}
\begin{verbatim}
NewTP version 0.1
=================

This is a short guide to compiling and using the NewTP
prototype implementation.

1. Requirements
---------------

The prototype is designed to run on the Linux operating
system. Compatibility with other operating systems is not
guaranteed.

To compile and run the prototype, you must have the
following software:

 * a C compiler
 * GNU-compatible Make program
 * Python 3
 * pkg-config
 * GNU SASL Library (gsasl) version 1.6 or higher
 * GnuTLS version 3.0 or higher
 * FUSE version 2.6 or higher

2. Compiling
------------

Follow these steps to compile the software:

1. Extract the source tarball
   $ tar xvjf newtp-0.1.tar.bz2
2. Go into the directory
   $ cd newtp-0.1
3. Start build process
   $ make

3. Running
----------

3.1 server
----------

usage: ./server [-p password] <shares>

If the -p argument is not given, server runs in anonymous mode.
Shares can be specified as follows:

 /path/to/share=name - this share is read-only
 -ro /path/to/share=name - this is also read-only
 -rw /path/to/share=name - this is read-write


3.2 client
----------

usage: ./client <hostname> <command> [path]

client can be invoked in one of three modes:

 - list contents of root directory:
   $ ./client <hostname> list
 - list contents of path:
   $ ./client <hostname> list /path/of/interest
 - download a remote file:
   $ ./client <hostname> get /path/to/fi.le


3.3 newfs
---------

usage: ./newfs <hostname> <mountpoint>

Mounts the filesystem exported on <hostname> onto the
directory <mountpoint>. To unmount, use the command:

fusermount -u <mountpoint>


\end{verbatim}
