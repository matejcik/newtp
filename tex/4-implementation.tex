% vim: tw=110:fo=n1lt

\chapter{Implementation}

As a proof of the proposed concept, we have implemented a NewTP server and two client applications. This
served to evaluate soundness of the design ideas, test out protocol features, and prove that the specification
does not place unnecessary complications on the implementation. This chapter gives a brief overview of the
software, describes how it is implemented, and proposes directions for future improvements.

The provided implementation is purposefully simplistic. In many cases, we opted for straightforward but
inefficient approaches in place of more involved techniques that would improve performance or robustness.
Please keep in mind that this is purely a prototype and it is not intended for production use. The software is
architected with further development in mind, but it should be considered a mere foundation for a more
efficient and reliable implementation.

The prototype only implements the core command set, not the extensions.

The software that we have developed runs on the Linux operating system. It consists of three executable
programs built from a common code base. The {\tt server} program is the NewTP server and the {\tt newfs}
program implements the FUSE filesystem module. Lastly, {\tt client} is a simple command-line tool for
downloading individual files.

\section{Common functionality}

Certain operations are required both for client and server implementations. All three programs share the
headers that define command, result and attribute codes, and headers with definitions of {\tt struct}s that
correspond to NewTP packet formats and arguments. The shared functionality, apart from utility functions,
consists of the following:
\begin{itemize}[nolistsep]
	\item Packing and unpacking data, specified by a format string, to and from buffers
	\item Auto-generated helper functions that can pack and unpack the corresponding {\tt struct}s
		directly
	\item Helper functions to set up and tear down a TLS session
	\item Helper functions to send and receive data from the network
	\item Wrapper functions for these helpers that cleanly close the connection and exit the program
		when a network error occurs
\end{itemize}

\section{Server}

The server program allows the user to specify directories to export through NewTP as ``shares'', that is, virtual
entries in the published filesystem root. The user can choose which of them should be read-only and which are
writable. It is also possible to enable password-based authentication.

The server is implemented in a traditional forking model. The main process binds to ports on all available
interfaces and listens for incoming connections. When a client connects, the main process spawns a child
process to handle the incoming connection, and resumes listening.

The child process first establishes the TLS encryption, processes NewTP session initialization and SASL
authentication, and enters an endless command-serving loop. In each iteration, the process blocks on reading
from the network until it has enough data to decode an incoming request header. After that, it retrieves the
request payload, stores it in a buffer, which is then handed over to the appropriate command procedure. This
procedure decodes request arguments if required, performs the command and places its results into an output
buffer.  When the command procedure returns, contents of the output buffer are sent as a reply to the client.

Core commands are implemented in a separate source file and do not perform any network communication -- that
is the sole responsibility of the main server code. This enforces the notion that commands should not have
``private conversations'' with the client. The only data shared between command invocations is placed in file
handle structures. This is used to retain open directory descriptors between {\tt READDIR} operations, and
perform the basic optimization of keeping file descriptors open between read and write operations.

File handle assignment is performed by another separate code file, so that its functionality can be easily
exposed to extensions. This code is responsible for resolving the share-based virtual paths to actual
filesystem paths on the server. There are some provisions for the ability to dynamically add or remove shares
at run-time, but this functionality is not implemented in the prototype. At the moment, the coupling between
the file handle module and the core command module is still relatively tight. This should be rectified in
future versions.

Command procedures share the same basic structure, although it is more complicated esp. in cases where there
are more system calls in sequence ({\tt WRITE}) or more complicated processing is required on the result data
({\tt READDIR}). The general structure is as follows:
\begin{enumerate}[nolistsep]
	\item Obtain the file handle
	\item Perform a system call
	\item Resolve error code to the appropriate NewTP result code
	\item Store results in output buffer
\end{enumerate}

\section{Filesystem module client}

The primary client implementation of NewTP is a module for the Linux FUSE filesystem. It connects to a server
and exposes its contents through the Linux virtual file system interface, so that all programs running on
the system can access it transparently.

The FUSE filesystem was extremely helpful when developing the protocol, because it represents a convenient
user interface to individual filesystem operations. Once the basic {\tt STAT} and {\tt REWINDDIR}/{\tt
READDIR} operations were in place, we could use the UNIX command-line tools to test every feature of the
protocol separately. Furthermore, running more complicated actions within the filesystem (such as rebuilding
the executables from source) allowed us to stress-test the implementation and easily discover missing
functionality.

The FUSE library itself was also very useful in implementing the module. It performs a great deal of error
checking, replaces unimplemented operations with stubs, and ensures that arguments passed to the client
program are well-formed and within acceptable ranges. Combined with the common functionality, this lets us
implement most operations in 5 to 10 lines of C code, and keep everything in a single source file. In fact, by
far the most complicated part of the code is a primitive hash-table implementation for file handle lookup.

Every operation procedure receives a path string and arguments for that particular operation. With that, the
procedure performs the following steps:
\begin{enumerate}[nolistsep]
	\item Look up or assign a file handle for this path string
	\item Build an appropriate NewTP command with arguments
	\item Send it over the network
	\item Block until reply arrives
	\item Resolve NewTP result code to appropriate error code
	\item Return results of the operation
\end{enumerate}

FUSE modules can run in one of two modes: single-threaded, where operations are serialized, and
multi-threaded, where more than one operation can run at once. To properly use the multi-threading capability,
we would have to maintain a background service thread to handle network communication and marshal requests and
replies. This would needlessly complicate the proof-of-concept implementation. Therefore, the {\tt newfs}
program runs in single-threaded mode.

This is obviously a huge performance hit. Combined with the fact that there is no caching, it means that every
filesystem operation waits for a network round-trip before allowing a different filesystem operation to
proceed. When running over the local loopback interface, the response time is acceptable for day-to-day file
operations, but it is not suited for much more than that.

The obvious direction for future improvements is multi-threaded operation. A marshal thread will be in charge
of maintaining the connection to the server and recovering from network errors.  Individual operations will
make blocking calls into this thread, it will send requests on their behalf and unblock the caller when the
corresponding reply arrives. The marshal thread would also be in a perfect position to manage application-wide
caches of files and directory entries, perform read-ahead requests, and respond to server notifications.

\section{Downloader client}

The {\tt client} program's main purpose is downloading files faster than {\tt newfs}. It was developed as
a scratch pad for implementation practices. Apart from downloading, it can also list directory contents
in {\tt ls}-like format. If developed further, it could become a general-purpose tool for working with NewTP
in shell scripts.

The implementation is very straightforward: the main procedure establishes a connection to the server, and
launches the appropriate procedure based on command line arguments. These procedures don't follow any
particular pattern. Each of them performs its task sequentially, prints results if appropriate, and exits the
program.

As stated above, the filesystem client waits for results of every read operation before issuing a new one. We
attempted to improve the performance of {\tt client} by sending read operations in batches. At start of the
session, it sends eight consecutive read requests before it starts waiting for replies. Every time a reply
arrives, the program issues a new read request. This way, there should be outstanding requests and replies at
all times, so the network bandwidth is better utilized.

In informal testing, the batch approach increased performance over the naïve approach by a factor between 50
and 100. This proves that the concept has merit. However, when compared with a mature HTTP server,
our NewTP implementations still leave a lot to be desired.
